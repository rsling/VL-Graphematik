\section{Übersicht}

\begin{frame}
  {Übersicht}
  \onslide<+->
  \begin{itemize}[<+->]
    \item Wo stehen Kommata?
      \Zeile
    \item Doppelfunktion oder Monofunktion?
    \item Grenzfälle
      \Zeile
    \item Empirie | \textit{obwohl} und \textit{weil} mit V2
      \Zeile
    \item \citet{Schaefer2018b,SchaeferSayatz2016}
  \end{itemize}
\end{frame}

\section[Befund]{Deskriptiver Befund}

\begin{frame}
  {Aufzählung}
  \onslide<+->
  \onslide<+->
  \begin{exe}
    \ex \alert{Peter, Paul und Mary} gehen in den Zoo.
    \ex \alert{Unter, neben und über} dem Werkstück für genügend Freiraum achten.
    \ex \alert{Wandern, Schwimmen, Radfahren} -- Volkssport pur!
    \ex Die Verbindung erfolgt \alert{form-, kraft- oder stoff}schlüssig.
  \end{exe}
  \onslide<+->
  \Zeile
  Kommatierung ist hier so flexibel wie Koordinationsstrukturen eben sind.
\end{frame}

\begin{frame}
  {Sätze}
  \onslide<+->
  \onslide<+->
  \begin{exe}
    \ex
    \begin{xlist}
	  \ex \grau{Die Sonne geht unter, der Mond geht auf.}
	  \ex \grau{Die Sonne geht unter, und der Mond geht auf.}
    \end{xlist}
    \Zeile
    \ex Adrianna weiß, \alert{dass es gleich regnen} \orongsch{wird}.
    \ex Michelle geht, \alert{obwohl die Party erst} \orongsch{beginnt}.
    \ex Adrianne hilft der Kollegin, \alert{die nassgeregnet} \orongsch{wurde}.
    \Zeile
    \ex \tuerkis{Adrianna glaubt, die Regenwolken zu sehen.}
  \end{exe}
  \onslide<+->
  \Zeile
  Diese Satzkommas lassen sich gut auf eine syntaktische Domäne eingrenzen.
\end{frame}

\begin{frame}
  {Sonstiges}
  \onslide<+->
  \onslide<+->
  \begin{exe}
    \ex Adrianna, \alert{eine Kollegin}, wurde nassgeregnet.
    \ex Die, \alert{übrigens unsinnige}, Behauptung der Monofunktion\\
    wird kaum vertreten.
    \ex Michelle will den Dobermann aufnehmen, \alert{als Pflegestelle}.
    \ex \alert{Ja}, Michelle kennt Adrianna.
  \end{exe}
  \onslide<+->
  \Zeile
  Hat das Komma hier primär einen Intonationseffekt?
\end{frame}

\section[Erklärung]{Erklärungsansätze}

\begin{frame}
  {Syntax | Doppelfunktion (Primus usw.)}
  \onslide<+->
  \onslide<+->
  Gibt es überhaupt eine "`Theorie des Kommas"'?\\
  \onslide<+->
  \Zeile
  \begin{itemize}[<+->]
    \item Nein | Ziel: \alert{optimale Beschreibungen von Verteilungen}\\
	    \Zeile
    \item syntaktisch keine Gemeinsamkeit zwischen Koordination und Nebensatz
    \item \ldots\ \alert{aber beides auf jeden Fall rein syntaktisch definierte Grenzen!}
	    \Halbzeile
    \item \alert{Intonationsgrenzen?} --- ja, als Folge der syntaktischen Grenze
    \item aber \rot{viele Intonationsgrenzen ohne Komma}
  \end{itemize}
\end{frame}

\begin{frame}
  {Syntax | Bredels Monofunktion I}
  \onslide<+->
  \begin{itemize}[<+->]
	  \item Behauptung | \alert{Doppelfunktion "`nicht lernbar"'}
		  \Zeile
    \item Wie bitte?
	    \Halbzeile
	    \begin{itemize}[<+->]
		    \item \alert{Homonymie?}\\
			    \textit{Kiefer}, \textit{Schloss}, \textit{Bank}
			    \Halbzeile
		    \item \alert{Synkretismus?}\\
			    \textit{dieser}, \textit{Menschen}, \textit{laufen}
			    \Halbzeile
		    \item \alert{strukturelle Ambiguität?}\\
			    \textit{Scully beobachtet den Außerirdischen mit dem Teloskop.}
	    \end{itemize}
  \end{itemize}
\end{frame}

\begin{frame}
  {Syntax | Bredels Monofunktion II}
  \onslide<+->
  \begin{itemize}[<+->]
	  \item Komma markiert \alert{"`Grenze im Parsingprozess"'}
	  \item also "`Online-Funktion"' in der Syntaxverarbeitung
		  \Halbzeile
	  \item \rot{keine} zugrundeliegende Syntaxtheorie
	  \item \rot{keine} ausgearbeitete Verabreitungstheorie
	  \item beliebig \rot{allgemeine Beschreibung} = immer Monofunktion\\
		  \grau{Die Funktion jedes Wortes ist die sprachliche Kommunikation!}
  \end{itemize}
\end{frame}

\begin{frame}
  {Syntax | Bredels Monofunktion III}
  \onslide<+->
  \onslide<+->
  Die (Fremd-)Daten sind nicht falsch, nur die Schlussfolgerung!\\
  \Zeile
  \begin{itemize}[<+->]
	  \item ähnlich wie bei der NP-Kopf-Großschreibung \ldots
		  \begin{itemize}[<+->]
			  \item \alert{natür}lich markiert Komma Phrasengrenzen
			  \item \alert{natürlich} beim Parsen (Verarbeitung) wichtiges Indiz
				  \Halbzeile
			  \item \grau{Das steht bei den Psycholinguisten, die Bredel rezipiert.}
				  \Halbzeile
			  \item \rot{Aber das erklärt nicht die Verteilung von Kommata im Deutschen!}
		  \end{itemize}
  \end{itemize}
\end{frame}

\begin{frame}
  {Grenzfälle | Satzverbindungen}
  <++>
\end{frame}

\begin{frame}
  {Grenzfälle | Kontrollinfinitive}
  <++>
\end{frame}

\section[Empirie]{Funktion und Empirie | \citet{SchaeferSayatz2016}}

\begin{frame}
  {\textit{obwohl} und \textit{weil} mit V2}
  <++>
\end{frame}

\begin{frame}
  {Unabhängigkeit von Sätzen}
  <++>
\end{frame}

\begin{frame}
  {Empirischer Befund I | \textit{obwohl}}
  <++>
\end{frame}

\begin{frame}
  {Empirischer Befund II | \textit{weil}}
  <++>
\end{frame}

\begin{frame}
  {Empirischer Befund IIIa | Vergleichskonstrukte}
  <++>
\end{frame}

\begin{frame}
  {Empirischer Befund IIIb | Vergleichskonstrukte}
  <++>
\end{frame}

\begin{frame}
  {Interpretation | Grammatische Kategorien und Komma}
  <++>
\end{frame}

%\section{Ausblick}


