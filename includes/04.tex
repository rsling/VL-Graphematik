\section{Übersicht}

\begin{frame}
  {Übersicht}
  \pause
  \begin{itemize}[<+->]
    \item \alert{Segmente} als Einheiten der Phonetik\slash Phonologie
    \item nicht alle Segmente überall: \alert{Verteilungen}
    \item Endrand-Desonorisierung, r-Vokalisierung, \textit{ich}\slash\textit{ach}-Laute usw.\\
      und \alert{Ableitung} phonetischer Formen aus lexikalischen Formen
    \item längbare, betonbare und unbetonbare Vokale
      \Zeile
    \item \citet[Abschnitt~5.1]{Schaefer2018b}
    \item zusätzliche Literatur: \citet{Eisenberg2013a}
  \end{itemize}
\end{frame}

\begin{frame}
  {Erinnerung | der Kernwortschatz I}
  \pause
  Was war nochmal der Kernwortschatz?\\
  \Halbzeile
  \pause
  \begin{itemize}[<+->]
    \item Wörter, für die \alert{die weitreichenden Generalisierungen gelten}
    \item = Wörter und Wortklassen mit \alert{hoher Typenhäufigkeit}
    \item \rot{nicht} die "`häufigen Wörter"' (= Tokenhäufigkeit)
    \item \rot{nicht} die Erbwörter (aber Erbwörter meistens im Kern)
  \end{itemize}
\end{frame}

\begin{frame}
  {Erinnerung | der Kernwortschatz II}
  \pause
  Was war nochmal der Kernwortschatz?\\
  \Halbzeile
  \pause
  \begin{itemize}[<+->]
    \item Kern-Substantive: Einsilbler (im Plural Trochäus) oder Trochäus
    \item warum gerade Substantive so zentral?\\
      \alert{mit Abstand die mächtigste Wortklasse}
      \Halbzeile
    \item \rot{Missverständnis}: Kern\slash Peripherie klar abgegrenzt
    \item je höher die Typenhäufigkeit, desto kerniger
    \item periphere Wörter, Konstruktionen usw.\ \alert{nicht weniger grammatisch}
  \end{itemize}
\end{frame}


\section{Konsonantenzeichen}

\begin{frame}
  {Terminologie | Di- und Trigraphen}
  \onslide<+->
  \onslide<+->
  Dipgraphen | zwei Zeichen für ein Segment\\
  <ch> für [ç] bzw.\ [χ]\\
  \onslide<+->
  Was ist mit <pf>?
  \onslide<+->
  \Zeile
  Trigraphen | Drei Zeichen für ein Segment\\
  <sch> für [ʃ]
  \Zeile
  \onslide<+->
  In ihrer Distribution gekoppelte Zeichen?\\
  <qu> für [kv]?
\end{frame}

\begin{frame}
  {Das Inventar (Kern)}
  \onslide<+->
  \onslide<+->
  Unigraphen\\
  b c d f g h j k l m n p q r s t v w z\\
  \Zeile
  Digraphen\\
  ch pf ng \rot{qu}\\
  \Zeile
  Trigraphen\\
  sch\\
\end{frame}

\begin{frame}
  {Besondere Doppelkonsonanz}
  \onslide<+->
  \onslide<+->
\end{frame}

\begin{frame}
  {Ordnung total: die Konsonantenzeichen}
  \pause
  \centering
  \resizebox{0.375\textwidth}{!}{
    \begin{tabular}{lll}
      \toprule
      \textbf{Segment} & \textbf{Buchstabe(n)} & \textbf{Beispielwörter} \\
      \midrule
     p & p & \textit{Plan} \\
     b & b & \textit{Baum}, \textit{Trab} \\
     p͡f & pf & \textit{Pfad} \\
     f & f & \textit{Fahrt} \\
     v & w & \textit{Wand} \\
     m & m & \textit{Mus} \\
     t & t & \textit{Tau} \\
     d & d & \textit{Dach}, \textit{Bild}\\
     t͡s & z & \textit{Zeit} \\
     \rot{s} & \rot{s} & \textit{Los} \\
     \rot{z} & \rot{s} & \textit{Sau} \\
     ʃ & sch & \textit{Schiff} \\
     n & n & \textit{Not}, \textit{Klang} \\
     l & l & \textit{Lob} \\
     ç & ch & \textit{Blech}, \textit{Wacht} \\
     ʝ & j & \textit{Jahr} \\
     k & k & \textit{Kiel} \\
     g & g & \textit{Gans}, \textit{Weg}, \textit{König} \\
     ʁ & r & \textit{Ritt}, \textit{Tür} \\
     h & h & \textit{Herz} \\
      \bottomrule
    \end{tabular}
  }
\end{frame}

\begin{frame}
  {Invarianz der Konsonantenzeichen}
  \pause
  \alert{Wir schreiben, wie unsere zugrundeliegenden Formen aussehen.}\\
  \pause
  \Zeile
  \centering
  \resizebox{0.9\textwidth}{!}{
    \begin{tabular}{lp{0.15cm}lp{0.15cm}llp{0.15cm}llp{0.15cm}l}
      \toprule
      \textbf{zugr.} && \textbf{Buch-} && \multicolumn{2}{l}{\textbf{phonetische}}    && \multicolumn{2}{l}{\textbf{phonologische}} && \textbf{phonetische} \\
      \textbf{Segm.} && \textbf{stabe(n)} && \multicolumn{2}{l}{\textbf{Realisierungen}} && \multicolumn{2}{l}{\textbf{Schreibungen}}  && \textbf{Schreibung} \\
      \midrule
      b && b && ba͡ɔm & loːp && \textit{Baum} & \textit{Lob} && *\textit{Lop} \\
      d && d && daχ & ʁɪnt && \textit{Dach} & \textit{Rind} && *\textit{Rint} \\
      n && n && naχt & klaŋ && \textit{Nacht} & \textit{Klang} && *\textit{Klaŋ} \\
      ç && ch && lɪçt & vaχt && \textit{Licht} & \textit{Wacht} && *\textit{Waχt} \\
      g && g && gans & køːnɪç && \textit{Gans} & \textit{König} && *\textit{Könich} \\
      ʁ && r && ʁuːm & to͡ɐ && \textit{Ruhm} & \textit{Tor} && *\textit{Toe} \\
      \bottomrule
    \end{tabular}
  }
  \Zeile
  \pause
  \begin{itemize}[<+->]
    \item einige Substitutionsphänome (anlautendes /kv/ als \textit{qu} usw.)
    \item \alert{Das Problem mit den \textit{s}-Schreibungen wird noch gelöst!}
  \end{itemize}
\end{frame}


\section{Ausblick}

