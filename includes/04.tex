\section{Übersicht}

\begin{frame}
  {Übersicht}
  \pause
  \begin{itemize}[<+->]
    \item Erinnerung | Kernwortschatz
      \Halbzeile
    \item Inventar der Konsonantenzeichen im Kern
      \Halbzeile
    \item phonographisches Schreibprinzip
      \Halbzeile
    \item Phonologie und Graphematik
      \Zeile
    \item \citet[Kapitel~15]{Schaefer2018b}
  \end{itemize}
\end{frame}

\begin{frame}
  {Erinnerung | der Kernwortschatz I}
  \pause
  Was war nochmal der Kernwortschatz?\\
  \Halbzeile
  \pause
  \begin{itemize}[<+->]
    \item Wörter, für die \alert{die weitreichenden Generalisierungen gelten}
    \item = Wörter und Wortklassen mit \alert{hoher Typenhäufigkeit}
    \item \rot{nicht} die "`häufigen Wörter"' (= Tokenhäufigkeit)
    \item \rot{nicht} die Erbwörter (aber Erbwörter meistens im Kern)
  \end{itemize}
\end{frame}

\begin{frame}
  {Erinnerung | der Kernwortschatz II}
  \pause
  Was war nochmal der Kernwortschatz?\\
  \Halbzeile
  \pause
  \begin{itemize}[<+->]
    \item Kern-Substantive: Einsilbler (im Plural Trochäus) oder Trochäus
    \item warum gerade Substantive so zentral?\\
      \alert{mit Abstand die mächtigste Wortklasse}
      \Halbzeile
    \item \rot{Missverständnis}: Kern\slash Peripherie klar abgegrenzt
    \item je höher die Typenhäufigkeit, desto kerniger
    \item periphere Wörter, Konstruktionen usw.\ \alert{nicht weniger grammatisch}
  \end{itemize}
\end{frame}


\section{Konsonanten}

\begin{frame}
  {Terminologie | Di- und Trigraphen}
  \onslide<+->
  \begin{itemize}[<+->]
    \item Digraphen | zwei Zeichen für ein Segment\\
      \Halbzeile
      \alert{<ch>} für [ç] bzw.\ [χ]\\
      \onslide<+->
      \Halbzeile
      Was ist mit \alert{<pf>}?
    \Zeile
    \item Trigraphen | drei Zeichen für ein Segment\\
      \Halbzeile 
      \alert{<sch>} für [ʃ]
    \Zeile
    \item In ihrer Distribution gekoppelte Zeichen?\\
      \Halbzeile
      \alert{<qu>} für [kv]?
  \end{itemize}
\end{frame}

\begin{frame}
  {Das Inventar (Kern)}
  \onslide<+->
  \begin{itemize}[<+->]
    \item Unigraphen\\
    \Halbzeile
    \alert{k g t d p b}\\
    \onslide<+->
    \alert{z}\\
    \onslide<+->
    \alert{h r j s ß f v w}\\
    \onslide<+->
    \alert{n m l}\\
    \Viertelzeile
    \onslide<+->
    \rot{c q x ?}\\
    \Zeile
  \item Digraphen\\
    \Halbzeile 
    \alert{ng ch pf}
    \onslide<+->
    \rot{qu?}\\
    \Zeile
  \item Trigraphen und Tetragraphen\\
    \Halbzeile 
    \alert{sch tsch}
    \onslide<+->
    \rot{chs?}\\
  \end{itemize}
\end{frame}

\begin{frame}
  {Besondere Doppelkonsonanz}
  \onslide<+->
  \begin{itemize}[<+->]
    \item Reguläre Doppelkonsonanz\\
    \Halbzeile
    \alert{ck tt pp rr ss ff nn mm ll}\\
    \Zeile
    \item Besondere Doppelkonsonanz\\
    \Halbzeile
    \rot{gg dd bb}\\
    \Zeile
    \item Was ist eigentlich mit \alert{<tz>}?
  \end{itemize}
\end{frame}

\begin{frame}
  {Phonographisches Schreibprinzip}
  \onslide<+->
  \onslide<+->
  Versuch: "`Jedes Segment wird durch einen Graphen\\
  (ggf.\ Digraphen usw.) verschriftet."'\\
  \onslide<+->
  \Zeile
  \begin{exe}
    \ex{} [k] \alert{K}ind [g] \alert{G}enau
    \ex{} [t] \alert{T}ante [d] \alert{d}anke
    \ex{} [p] \alert{P}aar [b] \alert{B}ar 
    \ex{} [t͡s] \alert{Z}unge
    \ex{} [h] \alert{H}and [r] \alert{r}ot [j] \alert{j}ung [f] \alert{F}inger [w] \alert{W}anne
    \ex{} [n] \alert{N}ase [m] \alert{M}und [l] \alert{L}ippe
  \end{exe}
\end{frame}


\begin{frame}
  {Problem 1 | Endrand-Desonorisierung}
  \onslide<+->
  \onslide<+->
  \begin{exe}
    \ex Bu\alert{g} [k] --- Bu\alert{g}es [g]
    \ex Ba\alert{d} [t] --- Ba\alert{d}es [d]
    \ex Lo\alert{b} [p] --- Lo\alert{b}es [b]
    \ex bra\alert{v} [f] --- bra\alert{v}er [v]
    \Halbzeile
    \ex besonders: el\alert{f} [f] --- El\alert{f}er [v]
  \end{exe}
  \onslide<+->
  \Zeile
  \alert{Ein Graph} entspricht \alert{zwei Artikulationen}.\\
  \alert{stimmhaft -- stimmlos} \grau{je nach Position in der Silbe}
\end{frame}


\begin{frame}
  {Problem 2 | <ch>}
  \onslide<+->
  \onslide<+->
  \begin{exe}
    \ex schli\alert{ch} [ç]
    \ex Ba\alert{ch} [χ]
  \end{exe}
  \onslide<+->
  \Zeile
  \alert{Ein Graph} entspricht \alert{zwei Artikulationen}.\\
  \alert{Artikulation weiter vorne bzw.\ hinten} \grau{nach vorderen\slash nicht-vorderen Vokalen}
\end{frame}

\begin{frame}
  {Problem 3 | g-Spirantisierung}
  \onslide<+->
  \onslide<+->
  \begin{exe}
    \ex weni\alert{g} [ç]
    \ex weni\alert{g}er [g]
  \end{exe}
  \onslide<+->
  \Zeile
  \alert{Ein Graph} entspricht \alert{zwei Artikulationen}.\\
  \alert{Plosiv vs.\ Frikativ} \grau{je nach Umgebung (Silbenauslaut, vorangehendes /ɪ/)}
\end{frame}

\begin{frame}
  {Problem 4 | r-Vokalisierungen}
  \onslide<+->
  \onslide<+->
  \begin{exe}
    \ex Tie\alert{r} [ti͡ɐ] -- Tie\alert{r}e [ti͡əʁə]
    \ex Cho\alert{r} [ko͡ɐ] --- Chö\alert{r}e [køʁə]
    \ex kna\alert{rr} [kna͡ə] --- kna\alert{rr}en [knaʁən]
  \end{exe}
  \onslide<+->
  \Zeile
  \alert{Ein Graph} entspricht \alert{zwei Artikulationen}.\\
  \alert{[ʁ] oder [ə] bzw.\ [ɐ]} \grau{im Silbenanlaut- bzw.\ auslaut}
\end{frame}

\begin{frame}
  {Phonologie, nicht Phonetik}
  \onslide<+->
  \onslide<+->
  Alle genannten "`Ausnahmen"' zeigen \alert{phonologische Prozesse},\\
  also Anpassungen an Strukturbedingungen des Deutschen!\\
  \onslide<+->
  \Zeile
  Das phonographische Prinzip | Die \alert{(Konsonanten)graphen} entsprechen\\
  je einem \alert{zugrundeliegenden Segment}.
\end{frame}

\begin{frame}
  {Ordnung total: die Konsonantenzeichen}
  \pause
  \centering
  \resizebox{0.375\textwidth}{!}{
    \begin{tabular}{lll}
      \toprule
      \textbf{Segment} & \textbf{Buchstabe(n)} & \textbf{Beispielwörter} \\
      \midrule
     p & p & \textit{Plan} \\
     b & b & \textit{Baum}, \textit{Trab} \\
     p͡f & pf & \textit{Pfad} \\
     f & f & \textit{Fahrt} \\
     v & w & \textit{Wand} \\
     m & m & \textit{Mus} \\
     t & t & \textit{Tau} \\
     d & d & \textit{Dach}, \textit{Bild}\\
     t͡s & z & \textit{Zeit} \\
     \rot{s} & \rot{s} & \textit{Los} \\
     \rot{z} & \rot{s} & \textit{Sau} \\
     ʃ & sch & \textit{Schiff} \\
     n & n & \textit{Not}, \textit{Klang} \\
     l & l & \textit{Lob} \\
     ç & ch & \textit{Blech}, \textit{Wacht} \\
     ʝ & j & \textit{Jahr} \\
     k & k & \textit{Kiel} \\
     g & g & \textit{Gans}, \textit{Weg}, \textit{König} \\
     ʁ & r & \textit{Ritt}, \textit{Tür} \\
     h & h & \textit{Herz} \\
      \bottomrule
    \end{tabular}
  }
\end{frame}

\begin{frame}
  {Invarianz der Konsonantenzeichen}
  \pause
  \alert{Wir schreiben, wie unsere zugrundeliegenden Formen aussehen.}\\
  \pause
  \Zeile
  \centering
  \resizebox{0.9\textwidth}{!}{
    \begin{tabular}{lp{0.15cm}lp{0.15cm}llp{0.15cm}llp{0.15cm}l}
      \toprule
      \textbf{zugr.} && \textbf{Buch-} && \multicolumn{2}{l}{\textbf{phonetische}}    && \multicolumn{2}{l}{\textbf{phonologische}} && \textbf{phonetische} \\
      \textbf{Segm.} && \textbf{stabe(n)} && \multicolumn{2}{l}{\textbf{Realisierungen}} && \multicolumn{2}{l}{\textbf{Schreibungen}}  && \textbf{Schreibung} \\
      \midrule
      b && b && ba͡ɔm & loːp && \textit{Baum} & \textit{Lob} && *\textit{Lop} \\
      d && d && daχ & ʁɪnt && \textit{Dach} & \textit{Rind} && *\textit{Rint} \\
      n && n && naχt & klaŋ && \textit{Nacht} & \textit{Klang} && *\textit{Klaŋ} \\
      ç && ch && lɪçt & vaχt && \textit{Licht} & \textit{Wacht} && *\textit{Waχt} \\
      g && g && gans & køːnɪç && \textit{Gans} & \textit{König} && *\textit{Könich} \\
      ʁ && r && ʁuːm & to͡ɐ && \textit{Ruhm} & \textit{Tor} && *\textit{Toe} \\
      \bottomrule
    \end{tabular}
  }
  \Zeile
  \pause
  \begin{itemize}[<+->]
    \item einige Substitutionsphänome (anlautendes /kv/ als \textit{qu} usw.)
    \item \alert{Das Problem mit den \textit{s}-Schreibungen wird noch gelöst!}
  \end{itemize}
\end{frame}


\section{Ausblick}


\begin{frame}
  {Konsonanten und Vokale}
  \onslide<+->
  \begin{itemize}[<+->]
  \item das System der \alert{Vokalschreibungen}
    \Halbzeile
    \item Besonderheiten der Konsonantenschreibungen (im Kern)
    \Halbzeile
    \item Überblick über sogenannte Dehnungs- und Schärfungsschreibungen
    \Zeile
    \item \citet[Kapitel~15]{Schaefer2018b}
  \end{itemize}
\end{frame}
