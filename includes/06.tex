\section{Übersicht}

\begin{frame}
  {Übersicht}
  \onslide<+->
  \begin{itemize}[<+->]
    \item \citet{Schaefer2018b}
  \end{itemize}
\end{frame}
\section{Dehnung und Schärfung}

\begin{frame}
  {Das Kreuz mit der Dehnungsschreibung}
  \pause
  \begin{itemize}[<+->]
    \item Dehnungs-\textit{h} (\textit{Reh}, \textit{Pfahl}) oder Dehnungs-Doppelvokal (\textit{Saat}, \textit{Boot})
    \item speziell bei \textit{i} (dort fast immer): Dehnungs-\textit{e} (\textit{Knie}, \textit{Dieb})
      \Halbzeile
    \item \alert{weitgehend redundant} (erst recht im Kern)
    \item \alert{unsystematisch} (\textit{Lid}, \textit{Lied} usw.)
      \Halbzeile
    \item mangels Systematik: \alert{oft Erwerbsprobleme}
    \item \ldots denen kaum systematisch zu begenen ist
  \end{itemize}
\end{frame}

\begin{frame}
  {Das Faszinosum der Schärfungsschreibung}
  \pause
  Dehnungs-\slash Schärfungsschreibungen (Einsilbler\slash trochäischer Zweisilbler)\\
  \Zeile
  \pause
  \centering
  \resizebox{0.85\textwidth}{!}{
    \begin{tabular}{lllllllll}
      \toprule
      & & & \textbf{ɪ} & \textbf{ʊ} & \multicolumn{2}{l}{\LocStrutGrph\textbf{ɛ̆}} & \textbf{ɔ} & \textbf{ă} \\
      \midrule

      \multirow{4}{*}{\rotatebox{90}{\textbf{ungespannt}}}

      & \multirow{2}{*}{\rotatebox{90}{\textbf{offen}}}
      & \textbf{einsilb.}  & \textit{\Nono}  & \textit{\Nono}           & \multicolumn{2}{l}{\LocStrutGrph\textit{\Nono}}         & \textit{\Nono}        & \textit{\Nono}           \\
      && \textbf{zweisilb.}  & \textit{Li.\alert{pp}e} & \textit{Fu.\alert{tt}er}         & \multicolumn{2}{l}{\LocStrutGrph\textit{We.\alert{ck}e}}        & \textit{o.\alert{ff}en}       & \textit{wa.\alert{ck}er}         \\
        & \multirow{2}{*}{\rotatebox{90}{\textbf{gesch.}}}
        & \textbf{einsilb.}  & \textit{Ki\rot{nn}}   & \textit{Schu\rot{tt}}    & \multicolumn{2}{l}{\LocStrutGrph\textit{Be\rot{tt}}}           & \textit{Ro\rot{ck}}         & \textit{Wa\rot{tt}}            \\
        && \textbf{zweisilb.}  & \textit{Rin.de} & \textit{Wun.der}        & \multicolumn{2}{l}{\LocStrutGrph\textit{Wen.de}}        & \textit{pol.ter}      & \textit{Tan.te}          \\

      \midrule

      \multirow{4}{*}{\rotatebox{90}{\textbf{gespannt}}}

      & \multirow{2}{*}{\rotatebox{90}{\textbf{offen}}}
        & \textbf{einsilb.}  & \textit{Knie}   & \textit{Schuh}       & \textit{Schnee, Reh}  & \textit{zäh}          & \textit{roh}          & (\textit{da})            \\
      && \textbf{zweisilb.}  & \textit{Bie.ne} & \textit{Kuh.le, Schu.le} & \textit{we.nig}       & \textit{Äh.re, rä.kel} & \textit{oh.ne, O.fen} & \textit{Fah.ne, Spa.ten} \\

      & \multirow{2}{*}{\rotatebox{90}{\textbf{gesch.}}}
        & \textbf{einsilb.}  & \textit{lieb}  & \textit{Ruhm, Glut}      & \textit{Weg}          & \textit{spät}           & \textit{rot}          & \textit{Tat}             \\
      && \textbf{zweisilb.}  & (\textit{lieb.lich}) & (\textit{lug.te})   & (\textit{red.lich})   & (\textit{wähl.te})     & (\textit{brot.los})   & (\textit{rat.los})       \\

      \midrule
      & & & \textbf{i} & \textbf{u} & \textbf{e} & \textbf{ε} & \textbf{o} & \textbf{a} \\

      \bottomrule
    \end{tabular}
  }
  \Halbzeile\pause
  \begin{itemize}[<+->]
    \item \alert{Schärfungsschreibung im Trochäus nur nach ungespanntem Vokal\\
      in offener Silbe, wenn Anfangsrand der Zweitsilbe konsonantisch}
    \item (\ldots und im geschlossenen Einsilbler mit ungespannten Vokal)
  \end{itemize}
\end{frame}

\begin{frame}
  {Details und oft Übersehenes}
  \pause
  \begin{itemize}[<+->]
    \item \alert{Schärfungsschreibung = Silbengelenkschreibung}
    \item Aber warum dann im Einsilbler (\textit{Kinn}, \textit{Bett}, \textit{Rock})?
      \begin{itemize}[<+->]
        \item Siehe nächste Woche!
      \end{itemize}
      \Halbzeile
    \item Merke: Silbengelenkschreibung nur da, wo auch Silbengelenk:
      \begin{itemize}[<+->]
        \item \alert{zwischen Erst- und Zweitsilbe des Trochäus}
        \item \alert{nach ungespanntem} (=kurzem) \alert{Vokal}
      \end{itemize}
      \Halbzeile
    \item \alert{keine Schärfungsschreibung bei Di- und Trigraphen}
      \begin{itemize}[<+->]
        \item \textit{Esche} [ɛʃ̣ə], \textit{zischen} [t͡sɪʃ̣ən]
        \item \textit{Kachel} [kaχ̣əl], \textit{Zeche} [t͡sɛç̣ə]
        \item \textit{Kringel} [kʁɪŋ̣əl], \textit{Zunge} [t͡sʊŋ̣ə]
      \end{itemize}
      \Halbzeile
    \item \rot{Warum sind stimmhaften Obstruenten im Silbengelenk unmöglich?}
      \begin{itemize}[<+->]
        \item Obstruent auch im Endrand der Erstsilbe: \alert{Endrand-Desonorisierung}
        \item \textit{Kladde}, \textit{Robbe}, \textit{Bagger}, ?\textit{prasseln} [pʁazəln], *\textit{quivveln}
        \item \ldots \rot{nicht Kern} (fünf oder sechs Typen, alle niederdeutsch)
      \end{itemize}
  \end{itemize}
\end{frame}

\begin{frame}
  {Eszett: Warum ist mir das wichtig, und worum gehts?}
  \pause
  \begin{itemize}[<+->]
    \item Problem für manche Schreiber*innen
    \item theorieinterne deduktive Argumentation (= Wissenschaft)
    \item Eliminierung des zugrundeliegenden /s/
      \Halbzeile
    \item immerhin: erhebliche \alert{Systemstraffung} durch Orthographiereform!
      \Halbzeile
    \item Erinnerung: Verteilung von /s/ und /z/
      \begin{itemize}[<+->]
        \item Wortanfang: nur /z/ (\textit{Sog} [zoːk], niemals *[soːk])
        \item Wortauslaut: nur /s/ (\textit{Mus} [muːs], niemals *[muːz])
        \item \alert{im Wortinneren nach ungespanntem Vokal: nur /s/ (\textit{Masse} [maṣə])}
        \item \rot{im Wortinneren nach gespanntem Vokal:\\
            /s/ (\textit{Straße} [ʃtʁaːsə]) und /z/ (\textit{Hase} [haːzə])}
      \end{itemize}
  \end{itemize}
\end{frame}


\newcommand{\phopro}{\ensuremath{\Rightarrow}}

\begin{frame}
  {Analyse des Eszett}
  \pause
  \begin{itemize}[<+->]
    \item \alert{Alle Positionen bis auf die \textit{ß}-Umgebung sind herleitbar:}
      \begin{itemize}[<+->]
        \item Wortanlaut (\textit{Sog} [zoːk]): zugrundeliegendes /z/ bleibt [z]
        \item Wortauslaut (\textit{Mus} [muːs]): zugrundeliegendes /z/ würde sowieso [s]\\
          wegen Endrand-Desonorisierung
        \item Wortinneren nach ungespanntem Vokal (\textit{Masse} [maṣə]): \alert{Silbengelenk}\\
          immer stimmlos wegen Endranddesonorisierung (/măzə/ undenkbar)
      \end{itemize}
      \Halbzeile
    \item \alert{Bis hierhin brauchen wir noch kein zugrundeliegendes /s/!}
      \Halbzeile
    \item zugrundeliegendes /s/ \rot{nur für das Wortinnere nach gespanntem Vokal}\\
      \textit{Straße} [ʃtʁaːsə] gegenüber \textit{Hase} [haːzə]
    \item \alert{Und wenn statt /s/ einfach /zz/ zugrundeliegt?}
    \item \alert{Und wenn /zz/ mit \textit{ß} geschrieben wird?}
    \item also: \textit{Bußen} als /buzzən/ \phopro [buːssən]
  \end{itemize}
\end{frame}

\begin{frame}
  {Eszett-Silben und die anderen \textit{s}}
  \pause
  \centering
  {\footnotesize\textit{Busen}:}\hspace{1em}\scalebox{0.55}{%
    \begin{forest}
      for tree={s sep+=1em}
      [Phonologisches Wort, calign=first
        [Silbe, calign=last
          [Ar., ake
            [b]
          ]
          [Reim
            [Kern, ake
              [uː]
            ]
          ]
        ]
        [Silbe, calign=last
          [Ar., ake, baseline
            [z]
          ]
          [Reim, calign=first
            [Kern, ake
              [ə]
            ]
            [Er., ake
              [n]
            ]
          ]
        ]
      ]
    \end{forest}
  }~\pause\hspace{1.5em}{\footnotesize\textit{Bussen}:}\hspace{1em}\scalebox{0.55}{%
    \begin{forest}
      for tree={s sep+=1em}
      [Phonologisches Wort, calign=first
        [Silbe, calign=last
          [Ar., ake
            [b]
          ]
          [Reim, calign=first
            [Kern, ake
              [ʊ]
            ]
            [Er., ake, name=BusenEr]
          ]
        ]
        [Silbe, calign=last
          [Ar., ake, baseline
            [s]
            {\draw[-] (.north) -- (BusenEr.south);}
          ]
          [Reim, calign=first
            [Kern, ake
              [ə]
            ]
            [Er., ake
              [n]
            ]
          ]
        ]
      ]
    \end{forest}
  }\\
  \pause
  \Zeile
  {\footnotesize\textit{Bußen} mit \alert{Endranddesonorisierung} und \orongsch{Assimilation}:}\hspace{1em}\scalebox{0.55}{%  
    \begin{forest}
      for tree={s sep+=1em}
      [Phonologisches Wort, calign=first
        [Silbe, calign=last
          [Ar., ake
            [b]
          ]
          [Reim, calign=first
            [Kern, ake
              [uː]
            ]
            [Er., ake
              [\alert{\textbf{s}}]
            ]
          ]
        ]
        [Silbe, calign=last
          [Ar., ake, baseline
            [\orongsch{\textbf{s}}]
          ]
          [Reim, calign=first
            [Kern, ake
              [ə]
            ]
            [Er., ake
              [n]
            ]
          ]
        ]
      ]
    \end{forest}
  }
\end{frame}


\begin{frame}
  {Schritt für Schritt}
  \pause
  \begin{enumerate}[<+->]
    \item zugrundeliegende Form: \alert{/buzzən/}
    \item Silbifizierung \phopro \{buz\orongsch{.}zən\}
    \item Längung gespannter Vokale \phopro \{b\orongsch{uː}z.zən\}
    \item Endranddesonorisierung \phopro \{buː\orongsch{s}.zən\}
    \item Assimilation des Anfangsrands \phopro \alert{[buːs.}\orongsch{s}\alert{ən]}
  \end{enumerate}
  \pause
  \begin{itemize}[<+->]
    \item Ist die Assimilation ein Taschenspielertrick?
    \item Nein, denn sie findet auch in anderen Fällen statt!
  \end{itemize}
  \pause
  \begin{exe}
    \ex\label{ex:dehnungsundschaerfungsschreibungen024}
    \begin{xlist}
      \ex{\label{ex:dehnungsundschaerfungsschreibungen025} /ɛ̆k\alert{z}ə/ \phopro\ [ʔɛk.\orongsch{s}ə] (\textit{Echse})}
      \pause
      \ex{\label{ex:dehnungsundschaerfungsschreibungen026} /ɛ̆ʁb\alert{z}e/ \phopro\ [ʔɛ͡əp.\orongsch{s}ə] (\textit{Erbse})}
    \end{xlist}
  \end{exe}
  \pause
  \begin{itemize}[<+->]
    \item Also ist das Konsonantenzeichen \textit{s} \rot{nicht} doppelt belegt.
    \item \alert{Es gibt zugrundeliegend nur /z/.}
  \end{itemize}
\end{frame}


\begin{frame}
  {Achtung: Grenz-\textit{h}: weder Dehnung noch Segment}
  \pause
  \begin{exe}
    \ex wehe /veə/
    \pause
    \ex Ruhe /ʁuə/
    \pause
    \ex fliehe /fliə/
    \pause
    \ex Krähe /kʁɛə/
  \end{exe}
  \pause
  \begin{itemize}[<+->]
    \item keine Dehnungsschreibung, siehe \textit{fliehe}
    \item \alert{Silbengrenzenanzeiger} zwischen Vokalen
      \Halbzeile
    \item Ausnahme: nach Diphthong steht Grenz-\textit{h} nicht\\
      (\textit{Reue}, \textit{Kleie}, \textit{Schreie}, \textit{Säue})
    \item bis auf Ausnahmen (\textit{verzeihen}, \textit{leihen}, \textit{Reihe}, \textit{Weiher})
  \end{itemize}
\end{frame}


\section{Ausblick}


