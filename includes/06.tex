\section{Übersicht}

\begin{frame}
  {Übersicht}
  \onslide<+->
  \begin{itemize}[<+->]
    \item \citet{Schaefer2018b}
  \end{itemize}
\end{frame}

\section{Silben}



\section{Dehnung und Schärfung}

\begin{frame}
  {Das Kreuz mit der Dehnungsschreibung}
  \pause
  \begin{itemize}[<+->]
    \item Dehnungs-\textit{h} (\textit{Reh}, \textit{Pfahl}) oder Dehnungs-Doppelvokal (\textit{Saat}, \textit{Boot})
    \item speziell bei \textit{i} (dort fast immer): Dehnungs-\textit{e} (\textit{Knie}, \textit{Dieb})
      \Halbzeile
    \item \alert{weitgehend redundant} (erst recht im Kern)
    \item \alert{unsystematisch} (\textit{Lid}, \textit{Lied} usw.)
      \Halbzeile
    \item mangels Systematik: \alert{oft Erwerbsprobleme}
    \item \ldots denen kaum systematisch zu begenen ist
  \end{itemize}
\end{frame}

\begin{frame}
  {Das Faszinosum der Schärfungsschreibung}
  \pause
  Dehnungs-\slash Schärfungsschreibungen (Einsilbler\slash trochäischer Zweisilbler)\\
  \Zeile
  \pause
  \centering
  \resizebox{0.85\textwidth}{!}{
    \begin{tabular}{lllllllll}
      \toprule
      & & & \textbf{ɪ} & \textbf{ʊ} & \multicolumn{2}{l}{\LocStrutGrph\textbf{ɛ̆}} & \textbf{ɔ} & \textbf{ă} \\
      \midrule

      \multirow{4}{*}{\rotatebox{90}{\textbf{ungespannt}}}

      & \multirow{2}{*}{\rotatebox{90}{\textbf{offen}}}
      & \textbf{einsilb.}  & \textit{\Nono}  & \textit{\Nono}           & \multicolumn{2}{l}{\LocStrutGrph\textit{\Nono}}         & \textit{\Nono}        & \textit{\Nono}           \\
      && \textbf{zweisilb.}  & \textit{Li.\alert{pp}e} & \textit{Fu.\alert{tt}er}         & \multicolumn{2}{l}{\LocStrutGrph\textit{We.\alert{ck}e}}        & \textit{o.\alert{ff}en}       & \textit{wa.\alert{ck}er}         \\
        & \multirow{2}{*}{\rotatebox{90}{\textbf{gesch.}}}
        & \textbf{einsilb.}  & \textit{Ki\rot{nn}}   & \textit{Schu\rot{tt}}    & \multicolumn{2}{l}{\LocStrutGrph\textit{Be\rot{tt}}}           & \textit{Ro\rot{ck}}         & \textit{Wa\rot{tt}}            \\
        && \textbf{zweisilb.}  & \textit{Rin.de} & \textit{Wun.der}        & \multicolumn{2}{l}{\LocStrutGrph\textit{Wen.de}}        & \textit{pol.ter}      & \textit{Tan.te}          \\

      \midrule

      \multirow{4}{*}{\rotatebox{90}{\textbf{gespannt}}}

      & \multirow{2}{*}{\rotatebox{90}{\textbf{offen}}}
        & \textbf{einsilb.}  & \textit{Knie}   & \textit{Schuh}       & \textit{Schnee, Reh}  & \textit{zäh}          & \textit{roh}          & (\textit{da})            \\
      && \textbf{zweisilb.}  & \textit{Bie.ne} & \textit{Kuh.le, Schu.le} & \textit{we.nig}       & \textit{Äh.re, rä.kel} & \textit{oh.ne, O.fen} & \textit{Fah.ne, Spa.ten} \\

      & \multirow{2}{*}{\rotatebox{90}{\textbf{gesch.}}}
        & \textbf{einsilb.}  & \textit{lieb}  & \textit{Ruhm, Glut}      & \textit{Weg}          & \textit{spät}           & \textit{rot}          & \textit{Tat}             \\
      && \textbf{zweisilb.}  & (\textit{lieb.lich}) & (\textit{lug.te})   & (\textit{red.lich})   & (\textit{wähl.te})     & (\textit{brot.los})   & (\textit{rat.los})       \\

      \midrule
      & & & \textbf{i} & \textbf{u} & \textbf{e} & \textbf{ε} & \textbf{o} & \textbf{a} \\

      \bottomrule
    \end{tabular}
  }
  \Halbzeile\pause
  \begin{itemize}[<+->]
    \item \alert{Schärfungsschreibung im Trochäus nur nach ungespanntem Vokal\\
      in offener Silbe, wenn Anfangsrand der Zweitsilbe konsonantisch}
    \item (\ldots und im geschlossenen Einsilbler mit ungespannten Vokal)
  \end{itemize}
\end{frame}

\begin{frame}
  {Details und oft Übersehenes}
  \pause
  \begin{itemize}[<+->]
    \item \alert{Schärfungsschreibung = Silbengelenkschreibung}
    \item Aber warum dann im Einsilbler (\textit{Kinn}, \textit{Bett}, \textit{Rock})?
      \begin{itemize}[<+->]
        \item Siehe nächste Woche!
      \end{itemize}
      \Halbzeile
    \item Merke: Silbengelenkschreibung nur da, wo auch Silbengelenk:
      \begin{itemize}[<+->]
        \item \alert{zwischen Erst- und Zweitsilbe des Trochäus}
        \item \alert{nach ungespanntem} (=kurzem) \alert{Vokal}
      \end{itemize}
      \Halbzeile
    \item \alert{keine Schärfungsschreibung bei Di- und Trigraphen}
      \begin{itemize}[<+->]
        \item \textit{Esche} [ɛʃ̣ə], \textit{zischen} [t͡sɪʃ̣ən]
        \item \textit{Kachel} [kaχ̣əl], \textit{Zeche} [t͡sɛç̣ə]
        \item \textit{Kringel} [kʁɪŋ̣əl], \textit{Zunge} [t͡sʊŋ̣ə]
      \end{itemize}
      \Halbzeile
    \item \rot{Warum sind stimmhaften Obstruenten im Silbengelenk unmöglich?}
      \begin{itemize}[<+->]
        \item Obstruent auch im Endrand der Erstsilbe: \alert{Endrand-Desonorisierung}
        \item \textit{Kladde}, \textit{Robbe}, \textit{Bagger}, ?\textit{prasseln} [pʁazəln], *\textit{quivveln}
        \item \ldots \rot{nicht Kern} (fünf oder sechs Typen, alle niederdeutsch)
      \end{itemize}
  \end{itemize}
\end{frame}

\begin{frame}
  {Eszett: Warum ist mir das wichtig, und worum gehts?}
  \pause
  \begin{itemize}[<+->]
    \item Problem für manche Schreiber*innen
    \item theorieinterne deduktive Argumentation (= Wissenschaft)
    \item Eliminierung des zugrundeliegenden /s/
      \Halbzeile
    \item immerhin: erhebliche \alert{Systemstraffung} durch Orthographiereform!
      \Halbzeile
    \item Erinnerung: Verteilung von /s/ und /z/
      \begin{itemize}[<+->]
        \item Wortanfang: nur /z/ (\textit{Sog} [zoːk], niemals *[soːk])
        \item Wortauslaut: nur /s/ (\textit{Mus} [muːs], niemals *[muːz])
        \item \alert{im Wortinneren nach ungespanntem Vokal: nur /s/ (\textit{Masse} [maṣə])}
        \item \rot{im Wortinneren nach gespanntem Vokal:\\
            /s/ (\textit{Straße} [ʃtʁaːsə]) und /z/ (\textit{Hase} [haːzə])}
      \end{itemize}
  \end{itemize}
\end{frame}




\begin{frame}
  {Achtung: Grenz-\textit{h}: weder Dehnung noch Segment}
  \pause
  \begin{exe}
    \ex wehe /veə/
    \pause
    \ex Ruhe /ʁuə/
    \pause
    \ex fliehe /fliə/
    \pause
    \ex Krähe /kʁɛə/
  \end{exe}
  \pause
  \begin{itemize}[<+->]
    \item keine Dehnungsschreibung, siehe \textit{fliehe}
    \item \alert{Silbengrenzenanzeiger} zwischen Vokalen
      \Halbzeile
    \item Ausnahme: nach Diphthong steht Grenz-\textit{h} nicht\\
      (\textit{Reue}, \textit{Kleie}, \textit{Schreie}, \textit{Säue})
    \item bis auf Ausnahmen (\textit{verzeihen}, \textit{leihen}, \textit{Reihe}, \textit{Weiher})
  \end{itemize}
\end{frame}


\section{Ausblick}


