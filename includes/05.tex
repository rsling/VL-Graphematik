\section{Übersicht}

\begin{frame}
  {Übersicht}
  \onslide<+->
  \begin{itemize}[<+->]
    \item Vokale im Kernwortschatz
    \item Vokale in der Peripherie
      \Halbzeile
    \item System der Vokalzeichen
    \item Ausblick Dehnungsschreibungen
    \item System der Diphthongschreibungen
      \Halbzeile
    \item \citet[Kapitel~15]{Schaefer2018b}
  \end{itemize}
\end{frame}

\section{Gespanntheit}

\begin{frame}[fragile]
  {Gespanntheit}
  \pause
  \begin{center}
    \resizebox{0.6\textwidth}{!}{
      \begin{tikzpicture}[scale=3.5,baseline=default]
        \large
        \tikzset{
        vowel/.style={fill=white, anchor=mid, text depth=0ex, text height=1ex},
        vowelgespannt/.style={circle,fill=gray!30, anchor=mid, text depth=0ex, text height=1ex,minimum size=4ex},
        dot/.style={circle,fill=black,minimum size=0.4ex,inner sep=0pt,outer sep=-1pt},
        }

        \coordinate (hf) at (0,2); % high front
        \coordinate (hb) at (2,2); % high back
        \coordinate (lf) at (1,0); % low front
        \coordinate (lb) at (2,0); % low back
        \def\V(#1,#2){barycentric cs:hf={(3-#1)*(2-#2)},hb={(3-#1)*#2},lf={#1*(2-#2)},lb={#1*#2}}

        % Chart key (vorne -- hinten).
        \draw [{Latex[round]}-] (\V (-.25,0)) -- (\V (-.25,.5))  node [above left] {\footnotesize vorne};
        \draw [-{Latex[round]}] (\V (-.25,1.5)) -- (\V (-.25,2)) node [above left] {\footnotesize hinten};
        \path (\V (-.25,1)) node[above] {\footnotesize zentral};

        % Chart key (hoch--tief).
        \draw [{Latex[round]}-] (\V (0,-.25)) -- +(270:.5cm)  node [above right,rotate=90] (vokaltrapez1) {\footnotesize hoch};
        \draw [{Latex[round]}-] (\V (3,-2.5)) -- +(270:-.5cm) node [above left,rotate=90] (vokaltrapez2) {\footnotesize tief};
        \path (\V (1.5,-1)) node[above,rotate=90] {\footnotesize mittel};

        % Grid.
        \draw [gray,thick] (\V(0,0)) -- (\V(0,2));
        \draw [gray,thick] (\V(3,0)) -- (\V(3,2));
        \draw [gray,thick] (\V(0,0)) -- (\V(3,0));
        \draw [gray,thick] (\V(0,2)) -- (\V(3,2));

        \path (\V(0,0))      node[vowelgespannt] (i)   {i};
        \path (\V(0.25,0))   node[vowelgespannt] (y)   {y};
        \path (\V(0.4,0.5))  node[vowel]         (ii)  {ɪ};
        \path (\V(0.65,0.5)) node[vowel]         (yy)  {ʏ};
        \path (\V(1,0))      node[vowelgespannt] (e)   {e};
        \path (\V(1.25,0))   node[vowelgespannt] (oe)  {ø};
        \path (\V(2,0))      node[vowelgespannt] (ee)  {ɛ};
        \path (\V(1.4,0.7))  node[vowel]         (eee) {ɛ̆};
        \path (\V(1.65,0.7)) node[vowel]         (oee) {œ};
        \path (\V(3,1))      node[vowelgespannt] (a)   {a};
        \path (\V(2.5,1))    node[vowel]         (aa)  {ă};
        \path (\V (1,2))     node[vowelgespannt] (o)   {o};
        \path (\V (1.5,1.4)) node[vowel]         (oo)  {ɔ};
        \path (\V (0,2))     node[vowelgespannt] (u)   {u};
        \path (\V (0.5,1.5)) node[vowel]         (uu)  {ʊ};

        \draw (i)  -- (ii);
        \draw (y)  -- (yy);
        \draw (e)  -- (eee);
        \draw (oe) -- (oee);
        \draw (ee) -- (eee);
        \draw (a)  -- (aa);
        \draw (o)  -- (oo);
        \draw (u)  -- (uu);
      \end{tikzpicture}
    }
  \end{center}
\end{frame}

\begin{frame}
  {Vokale im Kernwortschatz}
  \onslide<+->
  \onslide<+->
  \alert{Gespannt} \orongsch{$\rightarrow$ betont und lang}\\
  \Viertelzeile
  \onslide<+->
  \begin{exe}
    \ex \textit{Tüte} /t\alert{y}tə/ $\Rightarrow$ [ˈt\orongsch{yː}.tə]\\
    \ex \textit{Magen} /m\alert{a}gən/ $\Rightarrow$ [ˈm\orongsch{aː}.gən]\\
    \ex \textit{vermietete} /fəʁm\alert{i}tətə/ $\Rightarrow$ [fɐ.ˈm\orongsch{iː}.tə.tə]\\
    \ex \textit{weniger} /v\alert{e}nɪgəʁ/ $\Rightarrow$ [ˈv\orongsch{eː}.nɪ.gɐ]\\
  \end{exe}
  \onslide<+->
  \Zeile
  \alert{Ungespannt} | \gruen{betont oder unbetont} \orongsch{$\rightarrow$ kurz}\\
  \Viertelzeile
  \begin{exe}
    \ex \textit{Sitte} /z\alert{ɪ}tə/ $\Rightarrow$ [ˈz\orongsch{ɪ}ṭə]\\
    \ex \textit{untersetzt} /\tuerkis{ʊ}ntəʁz\alert{ɛ}t͡st/ $\Rightarrow$ [ʔʊn.tɐ.ˈz\orongsch{ɛ}t͡st]\\
    \ex \textit{motzte} /m\alert{ɔ}t͡stə/ $\Rightarrow$ [ˈm\orongsch{ɔ}t͡s.tə]\\
    \ex \textit{unglaublich} /\alert{ʊ}ngla͡ɔblɪç/ $\Rightarrow$ [ʔ\orongsch{ʊ}n.ˈgla͡ɔb.lɪç]\\
  \end{exe}
\end{frame}

\begin{frame}
  {Gespanntheit im Kernwortschatz}
  \onslide<+->
  \onslide<+->
  \large
  \alert{Im Kernwortschatz sind gespannte Vokale immer\\
  betont und lang.} Zu jedem gespannten Vokal gibt es\\
  einen entsprechenden ungespannten Vokal.\\
  Der ungespannte ist betont oder unbetont,\\
  aber immer kurz.\\
  \Zeile
  \onslide<+->
  Die Länge muss also nicht markiert werden, sondern folgt\\
  aus Betonung und Gespanntheit.\\
  \Zeile
  \onslide<+->
  Trochäus-Regel plus Morphologie machen außerdem\\
  den Akzentsitz vorhersagbar!
\end{frame}

\begin{frame}
  {Vorhersagbarkeit des Akzentsitzes I}
  \onslide<+->
  \onslide<+->
  Wieso Trochäus-Regel + Morphologie = Akzentsitz?\\
  \onslide<+->
  \Zeile
  \begin{itemize}[<+->]
    \item \orongsch{Simplex}
      \Halbzeile
    \begin{itemize}[<+->]
      \item \textit{Mut} /mut/ \ensuremath{\Rightarrow} [ˈmuːt]\\
        Im Kern-Einsilber-Stamm: Akzent auf der \alert{einen Silbe}
        \Halbzeile
      \item \textit{Mitte} /mɪte/ \ensuremath{\Rightarrow} [ˈmɪṭə]\\
        Im Kern-Zweisilbler-Stamm: \alert{Trochäus}
        \Halbzeile
      \item \textit{wenigere} /venɪgəʁə/ \ensuremath{\Rightarrow} [ˈveː.nɪ.gə.ʁə]\\
        In längeren Flexionsformen: \alert{Stammakzent} bleibt
    \end{itemize}
  \end{itemize}
\end{frame}


\begin{frame}
  {Vorhersagbarkeit des Akzentsitzes II}
  \onslide<+->
  \onslide<+->
  Wieso Trochäus-Regel + Morphologie = Akzentsitz?\\
  \onslide<+->
  \Halbzeile
  \begin{itemize}[<+->]
    \item \orongsch{Derivate}
    \begin{itemize}[<+->]
      \item \textit{be:end-en} /bəɛndən/ \ensuremath{\Rightarrow} [bə.\blau{ˈʔɛn}.dən]
      \item \textit{unter:scheid-en} /ʊntəʁʃa͡ɛdən/ \ensuremath{\Rightarrow} [ʔʊn.tɐ.\blau{ˈʃa͡ɛ}.dən]
      \item \textit{ge:leg-en} /gəlegən/ \ensuremath{\Rightarrow} [gə.\blau{ˈleː}.gən]
      \item \textit{Eigen:heit} /a͡ɛgənha͡ɛt/ \ensuremath{\Rightarrow} [\blau{ˈʔa͡ɛ}.gən.ha͡ɛt]
      \item \textit{umfahren} /ʊmfaʁən/ \ensuremath{\Rightarrow} [\gruen{ˈʔʊm}.faː.ʁən]
      \item \textit{Unterschied} /ʊntəʁʃid/ \ensuremath{\Rightarrow} [\rot{ˈʔʊn}.tɐ.ʃiːt]
      \item \textit{Faselei} /fazəla͡ɛ/ \ensuremath{\Rightarrow} [faː.zə.\orongsch{ˈla͡ɛ}]
        \Halbzeile
      \item Fast alle Affixe lassen den Akzent auf dem \blau{Stamm}.
      \item \gruen{Verbpartikeln} (nicht Verbpräfixe) ziehen den Akzent an.
      \item \rot{Verpräfixe} ziehen in der Nominalisierung ebenfalls den Akzent an.
      \item Wenige \orongsch{Affixe} ziehen den Akzent an.
    \end{itemize}
  \end{itemize}
\end{frame}

\begin{frame}
  {Vorhersagbarkeit des Akzentsitzes III}
  \onslide<+->
  \onslide<+->
  Wieso Trochäus-Regel + Morphologie = Akzentsitz?\\
  \onslide<+->
  \Halbzeile
  \begin{itemize}[<+->]
    \item \orongsch{Komposita}
    \begin{itemize}[<+->]
      \item \textit{Tankstelle} /tănkʃtɛlə/ \ensuremath{\Rightarrow} [\blau{ˈtaŋk}.ʃtɛḷə]
      \item \textit{Tankstellenwart} /tănkʃtɛlənvaʁt/ \ensuremath{\Rightarrow} [\blau{ˈtaŋk}.ʃtɛḷən.va͡ət]
      \item \textit{Tankstellenwartausbildung} /tănkʃtɛlənvaʁta͡ʊsbɪldʊng/ \ensuremath{\Rightarrow} [\blau{ˈtaŋk}.ʃtɛḷən.va͡ət.ʔa͡ɔs.bɪl.dʊŋ]
        \Halbzeile
      \item Der Akzent bleibt immer uf dem Erstglied.
      \item \grau{Nebenakzente liegen auf den anderen Gliedern.}
    \end{itemize}
  \end{itemize}
\end{frame}

\begin{frame}
  {Fremdwortschatz mit freiem Akzentsitz}
  \resizebox{0.7\textwidth}{!}{
  \begin{minipage}{\textwidth}
    \begin{tabular}{lll}
      \textit{Idee} & /\rot{i}d\gruen{ˈe}/ & [ʔ\rot{i}.ˈd\gruen{eː}]\\
      \textit{Initiative} & /\rot{i}n\rot{i}t͡sʝ\rot{a}t\gruen{ˈi}v\grau{ə}/ & [ʔ\rot{i}.n\rot{i}.t͡sʝ\rot{a}.ˈt\gruen{iː}.v\grau{ə}]\\
      \textit{inspirieren} & /\blau{ɪ}nsp\rot{i}ʁ\gruen{ˈi}ʁ\grau{ə}n/ & [ʔɪn.sp\rot{i}.ˈʁ\gruen{iː}.ʁ\grau{ə}n] \\
      
      \textit{Methyl} & /m\rot{e}t\gruen{ˈy}l/ & [m\rot{e}.ˈt\gruen{yː}l]\\
      \textit{Québec} & /k\rot{e}b\gruen{ˈɛ}k/ & [k\rot{e}.ˈbɛk]\\
      \textit{integriert} & /\blau{ɪ}nt\rot{e}gʁ\gruen{ˈi}ʁt/ & [ʔɪn.t\rot{e}.ˈgʁi͡ɐt]\\
      \textit{debattieren} & /d\rot{e}b\rot{a}t\gruen{ˈi}ʁ\grau{ə}n/ & [d\rot{e}.ba.ˈt\gruen{iː}.ʁ\grau{ə}n] \\
      
      \textit{Utopie} & /\rot{u}t\rot{o}p\gruen{ˈi}/ & [ʔ\rot{u}.to.ˈp\gruen{iː}]\\
      \textit{Uran} & /\rot{u}ʁ\gruen{ˈa}n/ & [ʔ\rot{u}.ˈʁ\gruen{aː}n] \\
      
      \textit{Motiv} & /m\rot{o}t\gruen{ˈi}v/ & [m\rot{o}.ˈt\gruen{iː}f]\\
      \textit{politisch} & /p\rot{o}l\gruen{ˈi}t\blau{ɪ}ʃ/ & [p\rot{o}.ˈl\gruen{iː}.t\blau{ɪ}ʃ]\\
      \textit{Phonologie} & /f\rot{o}n\rot{o}l\rot{o}g\gruen{ˈi}/ & [f\rot{o}.n\rot{o}.l\rot{o}.ˈg\gruen{iː}] \\
      
      \textit{Ökonomie} & /\rot{ø}k\rot{o}n\rot{o}m\gruen{ˈi}/ & [ʔ\rot{ø}.ko.no.ˈm\gruen{iː}]\\
      \textit{manövrieren} & /m\rot{a}n\rot{ø}vʁ\gruen{ˈi}ʁ\grau{ə}n/ & [ma.n\rot{ø}.ˈvʁ\gruen{iː}.ʁ\grau{ə}n] \\
      
      \textit{Büro} & /b\rot{y}r\gruen{ˈo}/ & [b\rot{y}.ˈʁ\gruen{oː}]\\
      \textit{Cuvée} & /k\rot{y}v\gruen{ˈe}/ & [k\rot{y}.ˈv\gruen{eː}] \\
    \end{tabular}
  \end{minipage}
  }\\
  \Halbzeile
  \onslide<+->
  \footnotesize \rot{gespannt + unbetont $\rightarrow$ kurz} | \gruen{gespannt + betont $\rightarrow$ lang} | \\
  \blau{ungespannt + kurz (betont oder unbetont)} | \grau{Schwa, immer unbetont und immer kurz}\\
  \Halbzeile
  Peripherie | Der einzige relevante Unterschied: \rot{Es gibt unbetonte gespannte\\
  (und damit kurze) Vokale.} Der Akzentsitz muss lexikalisch spezifiziert sein.\\
\end{frame}


\begin{frame}
  {Gespanntheit im erweiterten Wortschatz}
  \pause
  \Large
  \rot{Im erweiterten Wortschatz sind gespannte Vokale\\
  lang, wenn sie betont sind, und kurz, wenn sie \\
  unbetont sind.} Auch im erweiterten Wortschatz\\
  gibt es keine ungespannten langen Vokale.\\
\end{frame}

\section{Vokalzeichen}

\begin{frame}
  {Ordnung naja: Vokalzeichen}
  \pause
  \centering
  \scalebox{0.8}{%
    \begin{tabular}{lp{0.5cm}llp{0.25cm}ll}
      \toprule
      \multirow{2}{*}{\textbf{Buchstabe}} && \multicolumn{2}{l}{\textbf{Segment}} && \multicolumn{2}{l}{\textbf{Segment}} \\
       && \textbf{gespannt} & \textbf{Beispiel} && \textbf{ungespannt} & \textbf{Beispiel} \\
      \midrule
      i  && i  & \textit{Igel} && ɪ & \textit{Licht} \\
      ü  && y  & \textit{Rübe} && ʏ & \textit{Rücken} \\
      u  && u  & \textit{Mut} && ʊ & \textit{Butter} \\
      e  && e  & \textit{Mehl} && ɛ̆ & \textit{Bett} \\
      ö  && ø & \textit{Höhle} && œ & \textit{Löffel} \\
      o  && o  & \textit{Ofen} && ɔ & \textit{Motte} \\
      ä  && ɛ  & \textit{Gräte} && ɛ̆ & \textit{Säcke} \\
      a  && a  & \textit{Wal} && ă & \textit{Wall} \\
      \bottomrule
    \end{tabular}
  }
    \Zeile
    \pause
    \begin{itemize}[<+->]
      \item \alert{für gespannte\slash ungespannte Vokalpaare nur je ein Zeichen}
      \item außerdem \textit{e}→/ɛ̆/ und \textit{ä}→/ɛ̆/
      \item "`speter"'-Dialekte zusätzlich \textit{e}→/e/ und \textit{ä}→/e/
        \Halbzeile
      \item \alert{Diphthonge} brechen zusätzlich das phonematische Prinzip
    \end{itemize}
\end{frame}


\begin{frame}[fragile]
  {Gespanntheit in "`speter"'-Dialekten}
  \begin{center}
    \resizebox{0.6\textwidth}{!}{
      \begin{tikzpicture}[scale=3.5,baseline=default]
        \large
        \tikzset{
        vowel/.style={fill=white, anchor=mid, text depth=0ex, text height=1ex},
        vowelgespannt/.style={circle,fill=gray!30, anchor=mid, text depth=0ex, text height=1ex,minimum size=4ex},
        dot/.style={circle,fill=black,minimum size=0.4ex,inner sep=0pt,outer sep=-1pt},
        }

        \coordinate (hf) at (0,2); % high front
        \coordinate (hb) at (2,2); % high back
        \coordinate (lf) at (1,0); % low front
        \coordinate (lb) at (2,0); % low back
        \def\V(#1,#2){barycentric cs:hf={(3-#1)*(2-#2)},hb={(3-#1)*#2},lf={#1*(2-#2)},lb={#1*#2}}

        % Chart key (vorne -- hinten).
        \draw [{Latex[round]}-] (\V (-.25,0)) -- (\V (-.25,.5))  node [above left] {\footnotesize vorne};
        \draw [-{Latex[round]}] (\V (-.25,1.5)) -- (\V (-.25,2)) node [above left] {\footnotesize hinten};
        \path (\V (-.25,1)) node[above] {\footnotesize zentral};

        % Chart key (hoch--tief).
        \draw [{Latex[round]}-] (\V (0,-.25)) -- +(270:.5cm)  node [above right,rotate=90] (vokaltrapez1) {\footnotesize hoch};
        \draw [{Latex[round]}-] (\V (3,-2.5)) -- +(270:-.5cm) node [above left,rotate=90] (vokaltrapez2) {\footnotesize tief};
        \path (\V (1.5,-1)) node[above,rotate=90] {\footnotesize mittel};

        % Grid.
        \draw [gray,thick] (\V(0,0)) -- (\V(0,2));
        \draw [gray,thick] (\V(3,0)) -- (\V(3,2));
        \draw [gray,thick] (\V(0,0)) -- (\V(3,0));
        \draw [gray,thick] (\V(0,2)) -- (\V(3,2));

        \path (\V(0,0))      node[vowelgespannt] (i)   {i};
        \path (\V(0.25,0))   node[vowelgespannt] (y)   {y};
        \path (\V(0.4,0.5))  node[vowel]         (ii)  {ɪ};
        \path (\V(0.65,0.5)) node[vowel]         (yy)  {ʏ};
        \path (\V(1,0))      node[vowelgespannt] (e)   {e};
        \path (\V(1.25,0))   node[vowelgespannt] (oe)  {ø};
%        \path (\V(2,0))      node[vowelgespannt] (ee)  {ɛ};
        \path (\V(1.4,0.7))  node[vowel]         (eee) {ɛ̆};
        \path (\V(1.65,0.7)) node[vowel]         (oee) {œ};
        \path (\V(3,1))      node[vowelgespannt] (a)   {a};
        \path (\V(2.5,1))    node[vowel]         (aa)  {ă};
        \path (\V (1,2))     node[vowelgespannt] (o)   {o};
        \path (\V (1.5,1.4)) node[vowel]         (oo)  {ɔ};
        \path (\V (0,2))     node[vowelgespannt] (u)   {u};
        \path (\V (0.5,1.5)) node[vowel]         (uu)  {ʊ};

        \draw (i)  -- (ii);
        \draw (y)  -- (yy);
        \draw (e)  -- (eee);
        \draw (oe) -- (oee);
%        \draw (ee) -- (eee);
        \draw (a)  -- (aa);
        \draw (o)  -- (oo);
        \draw (u)  -- (uu);
      \end{tikzpicture}
    }
  \end{center}
\end{frame}




\begin{frame}
  {Gründe für das System der Vokalzeichen}
  \pause
  \begin{itemize}[<+->]
    \item im Kern: \alert{Kopplung von Gespanntheit, Länge und Betonung}
    \item aber trotzdem \alert{keine zugrundeliegenden Formen} für Gespanntheitspaare
    \item zusammen mit \alert{Silbengelenkschreibung} (s.\,u.) aber\\
      kaum Bedarf an graphematischer Differenzierung
      \Halbzeile
    \item außerdem Entwicklung von \alert{Dehnungsschreibungen}\\
      zur Desambiguierung
    \item \ldots weil \alert{Gespanntheit + Akzent → Länge}
      \Halbzeile
    \item trotzdem suboptimal
  \end{itemize}
\end{frame}

\begin{frame}
  {Realisierungen der Dehnungsschreibung}
  \onslide<+->
  \onslide<+->
  Gespanntheitsmarkierung |\\
  \alert{h}, \orongsch{nichts}, \grau{Doppelvokal} oder bei <i> die \alert{<ie>-Schreibung}\\
  \Zeile
  \begin{tabular}{llllll}
    /i/ &           \rot{*<ih>} & \alert{<ie>} & \orongsch{<i>} &          \rot{*<ii>} & \textit{R\alert{ie}men}, \textit{\orongsch{I}gel}, *\textit{Kn\rot{ii}b}, *\textit{Kn\rot{ih}p} \\
    /y/ & \whyte{*}\alert{<üh>} &              & \orongsch{<ü>} &          \rot{*<üü>} & \textit{B\alert{üh}ne}, \textit{m\orongsch{ü}de}, *\textit{B\rot{üü}ke} \\
    /e/ & \whyte{*}\alert{<eh>} &              & \orongsch{<e>} & \whyte{*}\grau{<ee>} & \textit{k\alert{eh}ren}, \textit{w\orongsch{e}nig}, \textit{S\grau{ee}} \\
    /ɛ/ & \whyte{*}\alert{<äh>} &              & \orongsch{<ä>} &          \rot{*<ää>} & \textit{\alert{Äh}re}, \textit{d\orongsch{ä}nisch}, *\textit{S\rot{ää}le} \\
    /ø/ & \whyte{*}\alert{<öh>} &              & \orongsch{<ö>} &          \rot{*<öö>} & \textit{st\alert{öh}nen}, \textit{fl\orongsch{ö}ten}, *\textit{d\rot{öö}fer} \\
    /u/ & \whyte{*}\alert{<uh>} &              & \orongsch{<u>} &          \rot{*<uu>} & \textit{K\alert{uh}le}, \textit{Sch\orongsch{u}le}, *\textit{Kr\rot{uu}fe} \\
    /o/ & \whyte{*}\alert{<oh>} &              & \orongsch{<o>} & \whyte{*}\grau{<oo>} & \textit{L\alert{oh}n}, \textit{B\orongsch{o}den}, \textit{d\grau{oo}f} \\
    /a/ & \whyte{*}\alert{<ah>} &              & \orongsch{<a>} & \whyte{*}\grau{<aa>} & \textit{W\alert{ah}n}, \textit{b\orongsch{a}den}, \textit{\grau{Aa}l} \\
  \end{tabular}\\
  \Zeile 
  \onslide<+->
  <i>, <u> und Umlautgraphen können nicht gedoppelt werden!\\
  Wir kommen zu den "`Dehnungsschreibungen"' noch ausführlich zurück.
\end{frame}

\begin{frame}
  {Diphthongschreibungen (Kern)}
  \onslide<+->
  \onslide<+->
  \begin{itemize}[<+->]
    \item Diphthonge als komplexe Einsegmente
    \item Diphthongzeichen damit \alert{Digraphen}
    \item Achtung | \rot{Lautwert im Diphthong ungleich Lautwert isoliert}
  \end{itemize}
  \onslide<+->
  \Zeile
  \begin{exe}
    \ex \textit{H\gruen{au}s} /h\alert{a͡ɔ}z/ $\rightarrow$ [ˈh\alert{a͡ɔ}s]
    \onslide<+->
    \ex 
    \begin{xlist}
      \ex \textit{M\gruen{ai}s} /m\alert{a͡ɛ}z/ $\rightarrow$ [ˈ\alert{ma͡ɛ}s] 
      \ex \textit{M\gruen{ei}se} /m\alert{a͡ɛ}ze/ $\rightarrow$ [ˈ\alert{ma͡ɛ}.zə] 
    \end{xlist}
    \onslide<+->
    \ex \begin{xlist}
      \ex \textit{Häuser} /h\alert{ɔ͡œ}zəʁ/ $\rightarrow$ [ˈh\alert{ɔ͡œ}.zɐ]
      \ex \textit{Schleuse} /ʃl\alert{ɔ͡œ}zə/ $\rightarrow$ [ˈʃl\alert{ɔ͡œ}.zə] 
    \end{xlist}
  \end{exe}
\end{frame}

\begin{frame}
  {System der Diphthongschreibungen?}
  \onslide<+->
  \onslide<+->
  \begin{center}
    \begin{tabular}{cc}
      \toprule
      \Large mögliche & \Large mögliche \\
      \Large Erstglieder & \Large Zweitglieder \\
      \midrule
      \Large \alert{a (ä) e} & \Large \gruen{i u} \\
      \bottomrule
    \end{tabular}
  \end{center}
  \Zeile
  \begin{itemize}[<+->]
    \item{ }<a> und <e> auch als Doppelvokale\\
      \textit{Haar}, \textit{Saat}, \textit{Waage}\\
      \textit{Beere}, \textit{leer}, \textit{Meer}
      \Halbzeile
    \item{ }<uu> und <ui> selbst in Phantasiewörtern ausgeschlossen\\
      *\textit{Diip}, *\textit{Kiibe}, *\textit{Duut}, *\textit{Kuute}
      \Zeile
    \item eindeutiges \alert{Diphthongsignal}: \alert{<i> und <u> nach Vokalzeichen}
  \end{itemize}
\end{frame}

\begin{frame}
  {Form (Graphie) der Vokalzeichen}
  \onslide<+->
  \onslide<+->
  Es gibt distributionell drei Gruppen von Vokalzeichen.\\
  \Zeile
  \begin{itemize}[<+->]
    \item \alert{<\textit{a}> <\textit{e}> <\textit{o}>}
      \begin{itemize}[<+->]
        \item typische Vokale \alert{ohne Oberlänge}
        \item \ldots und \alert{graphische Rundheit}
      \end{itemize}
      \Halbzeile
    \item \orongsch{<\textit{u}> <\textit{i}>}
      \begin{itemize}[<+->]
        \item partiell atypisch durch \orongsch{geringere graphische Rundheit}
        \item als Zweitglieder im Diphthong \orongsch{näher am Endrand (Coda)}\\
          (graphisch \orongsch{konsonantischer})
        \item \orongsch{nicht verdoppelbar} (Einfluss der Graphie?)
        \item{} <ie> Dehnungsschreibung mit prototypischen <e>-Graphen
      \end{itemize}
      \Halbzeile
    \item \rot{<\textit{ä}> <\textit{ö}> <\textit{ü}>}
      \begin{itemize}[<+->]
        \item atypische Vokale durch \rot{Oberlänge}
        \item \rot{nicht verdoppelbar} (Einfluss der Graphie?)
      \end{itemize}
  \end{itemize}
\end{frame}


\section{Ausblick}

\begin{frame}
  {Dehnung und Schärfung}
  \onslide<+->
  \begin{itemize}[<+->]
    \item Dehnungsschreibung | schwierig vorhersagbar
    \item Schärfungsschreibung | immer vorhersagbar
    \item scheinbare Dehnungsschreibungen
    \item Ist die Dehnungsschreibung verzichtbar?
      \Zeile
    \item Eszett
      \Zeile
    \item \citet[Kapitel~15]{Schaefer2018b}
  \end{itemize}
\end{frame}
