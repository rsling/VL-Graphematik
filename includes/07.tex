\section{Übersicht}

\begin{frame}
  {Übersicht}
  \onslide<+->
  \begin{itemize}[<+->]
    \item \citet{Schaefer2018b}
  \end{itemize}
\end{frame}


\begin{frame}
  {Das Kreuz mit der Dehnungsschreibung}
  \pause
  \begin{itemize}[<+->]
    \item Dehnungs-\textit{h} (\textit{Reh}, \textit{Pfahl}) oder Dehnungs-Doppelvokal (\textit{Saat}, \textit{Boot})
    \item speziell bei \textit{i} (dort fast immer): Dehnungs-\textit{e} (\textit{Knie}, \textit{Dieb})
      \Halbzeile
    \item \alert{weitgehend redundant} (erst recht im Kern)
    \item \alert{unsystematisch} (\textit{Lid}, \textit{Lied} usw.)
      \Halbzeile
    \item mangels Systematik: \alert{oft Erwerbsprobleme}
    \item \ldots denen kaum systematisch zu begenen ist
  \end{itemize}
\end{frame}



%%%%%%% ACHTUNG! Chaotisches Copy & Paste!


\begin{frame}
  {Warum reden wir jetzt gleich vom Silbengewicht?}
  \pause
  Wir erfassen zwei wesentliche Beobachtungen:
  \pause
  \Zeile
  \begin{itemize}[<+->]
    \item Es gibt u.\,a.\ Einschränkungen der Besetzungsmöglichkeiten\\
      des \alert{Endrands}, die von der \alert{Länge des Kern-Vokals} abhängen.
    \item Offene Silben mit kurzem Vokal gibt es (fast) nur mit Schwa.
  \Zeile
\item Diese Beschränkung betrifft also den \alert{Reim}.
  \end{itemize} 
\end{frame}

\begin{frame}
  {Silbengewicht als Beschränkung im Reim}
  \pause
  \Halbzeile
  \begin{center}
  \scalebox{0.8}{
  \begin{tabular}{llll}
    \toprule
                         & \textbf{Kern}        & \textbf{Endrand} & \textbf{Beispiele} \\
    \midrule
    \textbf{einmorig}    & \multirow{2}{*}{/ə/} & & \multirow{2}{*}{{}[ʔeː.ə], [tʁuː.ə]} \\
    (überleicht)         &                      & & \\
    \midrule
    \textbf{zweimorig}   & V & C & {}[ʔap], [knap]\\
    (leicht)             & VV & & {}[bla͡ɔ], [ʃneː], \rot{*[ʃne]} \\
    \midrule
    \textbf{dreimorig}   & V & CC & {}[balt], [ʔɪst], [nakt], \rot{*[baːlk]}, \rot{*[ʔiːmʃ]} \\
    (schwer)             & VV & C & {}[zoːk], [la͡ɔp], \rot{*[baːŋk]}, \rot{*[kvaːlm]} \\
    \bottomrule
  \end{tabular}
  }
  \end{center}
  \pause
  \Zeile
  \raggedright
  \begin{itemize}[<+->]
    \item \alert{Nur der \textbf{Reim} ist für das Silbengewicht relevant!}
    \item überleichte (einmorige) Silben nur mit Schwa\ldots\\
      und in speziellen Umgebungen (siehe unten, Korrektur zu EGBD3) \\
    \item überschwere (vier- oder mehrmorige) Silben \rot{niemals} möglich
  \end{itemize}
\end{frame}


\begin{frame}
  {Extrasilbisch II}
  \pause
  \scalebox{0.85}{\begin{minipage}{\textwidth} 
  \begin{exe}
    \ex Nicht überschwer (also max.\ drei Moren):
    \begin{xlist}
      \ex /ăçt/ $\Rightarrow$ [ʔaχt] (\textit{Acht})
      \pause
      \ex /lɛ̆st/ $\Rightarrow$ [lɛst] (\textit{lässt})
      \pause
      \ex /năkt/ $\Rightarrow$ [nakt] (\textit{nackt})
      \pause
      \ex /kʁăçs/ $\Rightarrow$ [kʁaχs] (\textit{Krachs})
      \pause
      \ex /ăçt/ $\Rightarrow$ [ʔaχt] (\textit{Acht})
    \end{xlist}
    \pause
    \ex Extrasilbizität wegen drohender Überschwere:
    \begin{xlist}
      \ex /lest/ $\Rightarrow$ [leːs+t] (\textit{lest})
      \pause
      \ex /ʁuft/ $\Rightarrow$ [ʁuːf+t] (\textit{ruft})
      \pause
      \ex /huts/ $\Rightarrow$ [huːt+s] (\textit{Huts})
      \pause
      \ex /legt/ $\Rightarrow$ [leːk+t] (\textit{legt})
      \pause
      \ex /la͡ɔfs/ $\Rightarrow$ [la͡ɔf+s] (\textit{Laufs})
      \pause
      \ex /fʊʁçt/ $\Rightarrow$ [fʊ͡əç+t] (\textit{Furcht})
      \pause
      \ex /fɛ̆lʃst/ $\Rightarrow$ [fɛlʃ+st] (\textit{fälschst})
    \end{xlist}
  \end{exe}
  \end{minipage}}
\end{frame}


\begin{frame}
  {Überleichte Silben mit betonbaren Vokalen?}
  \pause
  Was ist mit:
  \begin{itemize}[<+->]
    \item \rot{[bʊ]} in [ˈbʊ.tɐ]
    \item \rot{[ma]} in [ˈma.t͡ʃə]
    \item \rot{[klɪ]} in [ˈklɪ.ŋə]
  \end{itemize}
  \Zeile
  \centering
  \pause
  \rot{Sind das doch einmorige (überleichte) Silben mit Vollvokal?}\\
  \Zeile
  \pause
  \raggedright
  Dieser Silbentyp tritt nur auf:\\
  \begin{itemize}[<+->]
    \item \alert{in (scheinbar) offenen Silben} (sonst nicht überleicht)
    \item \alert{in der betonten Silbe eines Trochäus}
    \item \alert{vor simplexen Anfangsrändern}
  \end{itemize}
\end{frame}

\begin{frame}
  {Silbengelenke}
  \pause
  Lösung: Die Silben sind \alert{nicht überleicht}, \alert{der Konsonant\\
  an der Silbengrenze gehört zum Endrand der ersten und\\
zum Anfangsrand der zweiten Silbe}.\\
  \pause
  \Zeile
  \begin{center}
  \begin{forest}
    [Wort
      [Silbe, calign=last
        [Anfangsrand, ake
          [m]
        ]
        [Reim, calign=first
          [Kern, ake
            [ɪ]
          ]
          [Endrand, ake, name=ERBaum]
        ]
      ]
      [Silbe, calign=last
        [Anfangsrand, ake
          [t]
          {\draw[-] (.north) -- (ERBaum.south);}
        ]
        [Reim
          [Kern, ake
            [ə]
          ]
        ]
      ]
    ]
  \end{forest}
  \end{center}
\end{frame}

\begin{frame}
  {Silbengelenke}
  \begin{center}
  \SonDiag[4]{{m/\nas/0, ɪ/\vok/0, t/\plo/1, ə/\vok/0}}    
  \end{center}
\end{frame}









\section{Eszett}

\newcommand{\phopro}{\ensuremath{\Rightarrow}}

\begin{frame}
  {Analyse des Eszett}
  \pause
  \begin{itemize}[<+->]
    \item \alert{Alle Positionen bis auf die \textit{ß}-Umgebung sind herleitbar:}
      \begin{itemize}[<+->]
        \item Wortanlaut (\textit{Sog} [zoːk]): zugrundeliegendes /z/ bleibt [z]
        \item Wortauslaut (\textit{Mus} [muːs]): zugrundeliegendes /z/ würde sowieso [s]\\
          wegen Endrand-Desonorisierung
        \item Wortinneren nach ungespanntem Vokal (\textit{Masse} [maṣə]): \alert{Silbengelenk}\\
          immer stimmlos wegen Endranddesonorisierung (/măzə/ undenkbar)
      \end{itemize}
      \Halbzeile
    \item \alert{Bis hierhin brauchen wir noch kein zugrundeliegendes /s/!}
      \Halbzeile
    \item zugrundeliegendes /s/ \rot{nur für das Wortinnere nach gespanntem Vokal}\\
      \textit{Straße} [ʃtʁaːsə] gegenüber \textit{Hase} [haːzə]
    \item \alert{Und wenn statt /s/ einfach /zz/ zugrundeliegt?}
    \item \alert{Und wenn /zz/ mit \textit{ß} geschrieben wird?}
    \item also: \textit{Bußen} als /buzzən/ \phopro [buːssən]
  \end{itemize}
\end{frame}

\begin{frame}
  {Eszett-Silben und die anderen \textit{s}}
  \pause
  \centering
  {\footnotesize\textit{Busen}:}\hspace{1em}\scalebox{0.55}{%
    \begin{forest}
      for tree={s sep+=1em}
      [Phonologisches Wort, calign=first
        [Silbe, calign=last
          [Ar., ake
            [b]
          ]
          [Reim
            [Kern, ake
              [uː]
            ]
          ]
        ]
        [Silbe, calign=last
          [Ar., ake, baseline
            [z]
          ]
          [Reim, calign=first
            [Kern, ake
              [ə]
            ]
            [Er., ake
              [n]
            ]
          ]
        ]
      ]
    \end{forest}
  }~\pause\hspace{1.5em}{\footnotesize\textit{Bussen}:}\hspace{1em}\scalebox{0.55}{%
    \begin{forest}
      for tree={s sep+=1em}
      [Phonologisches Wort, calign=first
        [Silbe, calign=last
          [Ar., ake
            [b]
          ]
          [Reim, calign=first
            [Kern, ake
              [ʊ]
            ]
            [Er., ake, name=BusenEr]
          ]
        ]
        [Silbe, calign=last
          [Ar., ake, baseline
            [s]
            {\draw[-] (.north) -- (BusenEr.south);}
          ]
          [Reim, calign=first
            [Kern, ake
              [ə]
            ]
            [Er., ake
              [n]
            ]
          ]
        ]
      ]
    \end{forest}
  }\\
  \pause
  \Zeile
  {\footnotesize\textit{Bußen} mit \alert{Endranddesonorisierung} und \orongsch{Assimilation}:}\hspace{1em}\scalebox{0.55}{%  
    \begin{forest}
      for tree={s sep+=1em}
      [Phonologisches Wort, calign=first
        [Silbe, calign=last
          [Ar., ake
            [b]
          ]
          [Reim, calign=first
            [Kern, ake
              [uː]
            ]
            [Er., ake
              [\alert{\textbf{s}}]
            ]
          ]
        ]
        [Silbe, calign=last
          [Ar., ake, baseline
            [\orongsch{\textbf{s}}]
          ]
          [Reim, calign=first
            [Kern, ake
              [ə]
            ]
            [Er., ake
              [n]
            ]
          ]
        ]
      ]
    \end{forest}
  }
\end{frame}


\begin{frame}
  {Schritt für Schritt}
  \pause
  \begin{enumerate}[<+->]
    \item zugrundeliegende Form: \alert{/buzzən/}
    \item Silbifizierung \phopro \{buz\orongsch{.}zən\}
    \item Längung gespannter Vokale \phopro \{b\orongsch{uː}z.zən\}
    \item Endranddesonorisierung \phopro \{buː\orongsch{s}.zən\}
    \item Assimilation des Anfangsrands \phopro \alert{[buːs.}\orongsch{s}\alert{ən]}
  \end{enumerate}
  \pause
  \begin{itemize}[<+->]
    \item Ist die Assimilation ein Taschenspielertrick?
    \item Nein, denn sie findet auch in anderen Fällen statt!
  \end{itemize}
  \pause
  \begin{exe}
    \ex\label{ex:dehnungsundschaerfungsschreibungen024}
    \begin{xlist}
      \ex{\label{ex:dehnungsundschaerfungsschreibungen025} /ɛ̆k\alert{z}ə/ \phopro\ [ʔɛk.\orongsch{s}ə] (\textit{Echse})}
      \pause
      \ex{\label{ex:dehnungsundschaerfungsschreibungen026} /ɛ̆ʁb\alert{z}e/ \phopro\ [ʔɛ͡əp.\orongsch{s}ə] (\textit{Erbse})}
    \end{xlist}
  \end{exe}
  \pause
  \begin{itemize}[<+->]
    \item Also ist das Konsonantenzeichen \textit{s} \rot{nicht} doppelt belegt.
    \item \alert{Es gibt zugrundeliegend nur /z/.}
  \end{itemize}
\end{frame}


\section[Konstanz]{Konstantschreibung}

\begin{frame}
  {Zur Erinnerung: unerklärte Doppelkonsonanten}
  \pause
  \centering
  \resizebox{0.7\textwidth}{!}{
    \begin{tabular}{lllllllll}
      \toprule
      & & & \textbf{ɪ} & \textbf{ʊ} & \multicolumn{2}{l}{\LocStrutGrph\textbf{ɛ̆}} & \textbf{ɔ} & \textbf{ă} \\
      \midrule

      \multirow{4}{*}{\rotatebox{90}{\textbf{ungespannt}}}

      & \multirow{2}{*}{\rotatebox{90}{\textbf{offen}}}
      & \textbf{einsilb.}  & \textit{\Nono}  & \textit{\Nono}           & \multicolumn{2}{l}{\LocStrutGrph\textit{\Nono}}         & \textit{\Nono}        & \textit{\Nono}           \\
      && \textbf{zweisilb.}  & \textit{Li.\alert{pp}e} & \textit{Fu.\alert{tt}er}         & \multicolumn{2}{l}{\LocStrutGrph\textit{We.\alert{ck}e}}        & \textit{o.\alert{ff}en}       & \textit{wa.\alert{ck}er}         \\
        & \multirow{2}{*}{\rotatebox{90}{\textbf{gesch.}}}
        & \textbf{einsilb.}  & \textit{Ki\rot{nn}}   & \textit{Schu\rot{tt}}    & \multicolumn{2}{l}{\LocStrutGrph\textit{Be\rot{tt}}}           & \textit{Ro\rot{ck}}         & \textit{Wa\rot{tt}}            \\
        && \textbf{zweisilb.}  & \textit{Rin.de} & \textit{Wun.der}        & \multicolumn{2}{l}{\LocStrutGrph\textit{Wen.de}}        & \textit{pol.ter}      & \textit{Tan.te}          \\

      \midrule

      \multirow{4}{*}{\rotatebox{90}{\textbf{gespannt}}}

      & \multirow{2}{*}{\rotatebox{90}{\textbf{offen}}}
        & \textbf{einsilb.}  & \textit{Knie}   & \textit{Schuh}       & \textit{Schnee, Reh}  & \textit{zäh}          & \textit{roh}          & (\textit{da})            \\
      && \textbf{zweisilb.}  & \textit{Bie.ne} & \textit{Kuh.le, Schu.le} & \textit{we.nig}       & \textit{Äh.re, rä.kel} & \textit{oh.ne, O.fen} & \textit{Fah.ne, Spa.ten} \\

      & \multirow{2}{*}{\rotatebox{90}{\textbf{gesch.}}}
        & \textbf{einsilb.}  & \textit{lieb}  & \textit{Ruhm, Glut}      & \textit{Weg}          & \textit{spät}           & \textit{rot}          & \textit{Tat}             \\
      && \textbf{zweisilb.}  & (\textit{lieb.lich}) & (\textit{lug.te})   & (\textit{red.lich})   & (\textit{wähl.te})     & (\textit{brot.los})   & (\textit{rat.los})       \\

      \midrule
      & & & \textbf{i} & \textbf{u} & \textbf{e} & \textbf{ε} & \textbf{o} & \textbf{a} \\

      \bottomrule
    \end{tabular}
  }\\
  \pause
  \Viertelzeile
  \raggedright
  \begin{itemize}[<+->]
    \item Warum \textit{Kinn}, \textit{Schutt}, \textit{Bett}, \textit{Rock}, \textit{Wattes}?
    \item \alert{nicht unterlassbare Gelenkschreibungen}
      \begin{itemize}[<+->]
        \item \textit{die Ki\alert{nn}e}
        \item \textit{des Schu\alert{tt}es}
        \item \textit{die Be\alert{tt}en}
        \item \textit{die Rö\alert{ck}e}
      \end{itemize}
    \item \alert{Die Schreibungen eines Stamms einander angleichen!} Sonst:
      \begin{itemize}[<+->]
        \item \textit{*Kin --- Kinne}
        \item \textit{Schut --- Schutt}
        \item \textit{Bet --- Betten}
        \item \textit{Rok --- Röcke}
      \end{itemize}
  \end{itemize}
\end{frame}

\begin{frame}
  {Andere Konstantschreibungen}
  \pause
  \begin{itemize}[<+->]
    \item andere Wortklassen
      \begin{itemize}[<+->]
        \item \textit{*plat --- pla\rot{tt} --- pla\alert{tt}er}
        \item \textit{*as --- a\rot{ß} --- a\alert{ß}en}
        \item aber: \textit{las --- lasen}
        \item \textit{*schlizte --- schli\rot{tz}te --- schli\alert{tz}en}
      \end{itemize}
      \Halbzeile
    \item andere Phänomene (nicht Silbengelenk oder \textit{ß})
      \begin{itemize}[<+->]
        \item \textit{*gest --- ge\rot{h}st --- ge\alert{h}en}
        \item \textit{*siest --- sie\rot{h}st --- se\alert{h}en}
        \item \textit{*Reume --- R\rot{äu}me --- R\alert{au}m}
        \item \textit{*leuft --- l\rot{äu}ft --- l\alert{au}fen}
      \end{itemize}
  \end{itemize}
\end{frame}

\section[Prinzipien]{Schreibprinzipien}

\begin{frame}
  {Zusammenfassung der besprochenen Schreibprinzipien I}
  \pause
  Korrespondenzen zur Phonologie\\
  \Zeile
  \pause
  \begin{itemize}[<+->]
    \item \alert{phonologisches Schreibprinzip}
      \begin{itemize}[<+->]
        \item Konsonantenzeichen (inkl.\ Di- und Trigraphen)\\
          entsprechen 1:1 zugrundeliegenden Segmenten.
        \item Paare von zugrundeliegendem gespanntem und ungespanntem Vokal\\
          entsprechen jeweils nur einem Vokalzeichen 
      \end{itemize}
     \Zeile 
    \item \alert{Prinzip der Silbengelenkschreibung}
      \begin{itemize}[<+->]
        \item Silbengelenke werden durch Konsonantendopplung markiert.
        \item Für Di- und Trigraphen gilt dies nicht.
      \end{itemize}
  \end{itemize}
\end{frame}

\begin{frame}
  {Zusammenfassung der besprochenen Schreibprinzipien II}
  Korrespondenzen zur Morphosyntax\\
  \Zeile
  \pause
  \begin{itemize}[<+->]
    \item \alert{Prinzip der Konstantschreibung}
      \begin{itemize}[<+->]
        \item Die Formen eines lexikalischen Wortes werden so ähnlich geschrieben,\\
          wie es angesichts der anderen Prinzipien möglich ist.
      \end{itemize}
      \Zeile
    \item \grau{Prinzip der Spatienschreibung}
      \begin{itemize}[<+->]
        \item \grau{Syntaktische Wörter werden durch Spatium getrennt.}
        \item \grau{Zweifelsfälle dabei sind morphosyntaktisch, nicht graphematisch.}
      \end{itemize}
      \Zeile
    \item \grau{Prinzip der positionsunabhängige Majuskelschreibung}
      \begin{itemize}[<+->]
        \item \grau{Substantive werden positionsunabhängig\\
        mit einleitender Majuskel geschrieben.}
      \end{itemize}
  \end{itemize}
\end{frame}





\section{Ausblick}


