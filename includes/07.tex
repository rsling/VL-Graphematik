\section{Übersicht}

\begin{frame}
  {Übersicht}
  \onslide<+->
  \begin{itemize}[<+->]
    \item \citet{Schaefer2018b}
  \end{itemize}
\end{frame}

\section[Konstanz]{Konstantschreibung}

\begin{frame}
  {Zur Erinnerung: unerklärte Doppelkonsonanten}
  \pause
  \centering
  \resizebox{0.7\textwidth}{!}{
    \begin{tabular}{lllllllll}
      \toprule
      & & & \textbf{ɪ} & \textbf{ʊ} & \multicolumn{2}{l}{\LocStrutGrph\textbf{ɛ̆}} & \textbf{ɔ} & \textbf{ă} \\
      \midrule

      \multirow{4}{*}{\rotatebox{90}{\textbf{ungespannt}}}

      & \multirow{2}{*}{\rotatebox{90}{\textbf{offen}}}
      & \textbf{einsilb.}  & \textit{\Nono}  & \textit{\Nono}           & \multicolumn{2}{l}{\LocStrutGrph\textit{\Nono}}         & \textit{\Nono}        & \textit{\Nono}           \\
      && \textbf{zweisilb.}  & \textit{Li.\alert{pp}e} & \textit{Fu.\alert{tt}er}         & \multicolumn{2}{l}{\LocStrutGrph\textit{We.\alert{ck}e}}        & \textit{o.\alert{ff}en}       & \textit{wa.\alert{ck}er}         \\
        & \multirow{2}{*}{\rotatebox{90}{\textbf{gesch.}}}
        & \textbf{einsilb.}  & \textit{Ki\rot{nn}}   & \textit{Schu\rot{tt}}    & \multicolumn{2}{l}{\LocStrutGrph\textit{Be\rot{tt}}}           & \textit{Ro\rot{ck}}         & \textit{Wa\rot{tt}}            \\
        && \textbf{zweisilb.}  & \textit{Rin.de} & \textit{Wun.der}        & \multicolumn{2}{l}{\LocStrutGrph\textit{Wen.de}}        & \textit{pol.ter}      & \textit{Tan.te}          \\

      \midrule

      \multirow{4}{*}{\rotatebox{90}{\textbf{gespannt}}}

      & \multirow{2}{*}{\rotatebox{90}{\textbf{offen}}}
        & \textbf{einsilb.}  & \textit{Knie}   & \textit{Schuh}       & \textit{Schnee, Reh}  & \textit{zäh}          & \textit{roh}          & (\textit{da})            \\
      && \textbf{zweisilb.}  & \textit{Bie.ne} & \textit{Kuh.le, Schu.le} & \textit{we.nig}       & \textit{Äh.re, rä.kel} & \textit{oh.ne, O.fen} & \textit{Fah.ne, Spa.ten} \\

      & \multirow{2}{*}{\rotatebox{90}{\textbf{gesch.}}}
        & \textbf{einsilb.}  & \textit{lieb}  & \textit{Ruhm, Glut}      & \textit{Weg}          & \textit{spät}           & \textit{rot}          & \textit{Tat}             \\
      && \textbf{zweisilb.}  & (\textit{lieb.lich}) & (\textit{lug.te})   & (\textit{red.lich})   & (\textit{wähl.te})     & (\textit{brot.los})   & (\textit{rat.los})       \\

      \midrule
      & & & \textbf{i} & \textbf{u} & \textbf{e} & \textbf{ε} & \textbf{o} & \textbf{a} \\

      \bottomrule
    \end{tabular}
  }\\
  \pause
  \Viertelzeile
  \raggedright
  \begin{itemize}[<+->]
    \item Warum \textit{Kinn}, \textit{Schutt}, \textit{Bett}, \textit{Rock}, \textit{Wattes}?
    \item \alert{nicht unterlassbare Gelenkschreibungen}
      \begin{itemize}[<+->]
        \item \textit{die Ki\alert{nn}e}
        \item \textit{des Schu\alert{tt}es}
        \item \textit{die Be\alert{tt}en}
        \item \textit{die Rö\alert{ck}e}
      \end{itemize}
    \item \alert{Die Schreibungen eines Stamms einander angleichen!} Sonst:
      \begin{itemize}[<+->]
        \item \textit{*Kin --- Kinne}
        \item \textit{Schut --- Schutt}
        \item \textit{Bet --- Betten}
        \item \textit{Rok --- Röcke}
      \end{itemize}
  \end{itemize}
\end{frame}

\begin{frame}
  {Andere Konstantschreibungen}
  \pause
  \begin{itemize}[<+->]
    \item andere Wortklassen
      \begin{itemize}[<+->]
        \item \textit{*plat --- pla\rot{tt} --- pla\alert{tt}er}
        \item \textit{*as --- a\rot{ß} --- a\alert{ß}en}
        \item aber: \textit{las --- lasen}
        \item \textit{*schlizte --- schli\rot{tz}te --- schli\alert{tz}en}
      \end{itemize}
      \Halbzeile
    \item andere Phänomene (nicht Silbengelenk oder \textit{ß})
      \begin{itemize}[<+->]
        \item \textit{*gest --- ge\rot{h}st --- ge\alert{h}en}
        \item \textit{*siest --- sie\rot{h}st --- se\alert{h}en}
        \item \textit{*Reume --- R\rot{äu}me --- R\alert{au}m}
        \item \textit{*leuft --- l\rot{äu}ft --- l\alert{au}fen}
      \end{itemize}
  \end{itemize}
\end{frame}

\section[Prinzipien]{Schreibprinzipien}

\begin{frame}
  {Zusammenfassung der besprochenen Schreibprinzipien I}
  \pause
  Korrespondenzen zur Phonologie\\
  \Zeile
  \pause
  \begin{itemize}[<+->]
    \item \alert{phonologisches Schreibprinzip}
      \begin{itemize}[<+->]
        \item Konsonantenzeichen (inkl.\ Di- und Trigraphen)\\
          entsprechen 1:1 zugrundeliegenden Segmenten.
        \item Paare von zugrundeliegendem gespanntem und ungespanntem Vokal\\
          entsprechen jeweils nur einem Vokalzeichen 
      \end{itemize}
     \Zeile 
    \item \alert{Prinzip der Silbengelenkschreibung}
      \begin{itemize}[<+->]
        \item Silbengelenke werden durch Konsonantendopplung markiert.
        \item Für Di- und Trigraphen gilt dies nicht.
      \end{itemize}
  \end{itemize}
\end{frame}

\begin{frame}
  {Zusammenfassung der besprochenen Schreibprinzipien II}
  Korrespondenzen zur Morphosyntax\\
  \Zeile
  \pause
  \begin{itemize}[<+->]
    \item \alert{Prinzip der Konstantschreibung}
      \begin{itemize}[<+->]
        \item Die Formen eines lexikalischen Wortes werden so ähnlich geschrieben,\\
          wie es angesichts der anderen Prinzipien möglich ist.
      \end{itemize}
      \Zeile
    \item \grau{Prinzip der Spatienschreibung}
      \begin{itemize}[<+->]
        \item \grau{Syntaktische Wörter werden durch Spatium getrennt.}
        \item \grau{Zweifelsfälle dabei sind morphosyntaktisch, nicht graphematisch.}
      \end{itemize}
      \Zeile
    \item \grau{Prinzip der positionsunabhängige Majuskelschreibung}
      \begin{itemize}[<+->]
        \item \grau{Substantive werden positionsunabhängig\\
        mit einleitender Majuskel geschrieben.}
      \end{itemize}
  \end{itemize}
\end{frame}





\section{Ausblick}


