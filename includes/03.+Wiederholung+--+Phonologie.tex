\begin{frame}
  {Übersicht}
  \pause
  \begin{itemize}[<+->]
    \item \alert{Segmente} als Einheiten der Phonetik\slash Phonologie
    \item nicht alle Segmente überall: \alert{Verteilungen}
    \item Endrand-Desonorisierung, r-Vokalisierung, \textit{ich}\slash\textit{ach}-Laute usw.\\
      und \alert{Ableitung} phonetischer Formen aus lexikalischen Formen
    \item längbare, betonbare und unbetonbare Vokale
  \end{itemize}
\end{frame}


\begin{frame}
  {Segmente}
  \pause
  \begin{itemize}[<+->]
    \item Transkriptionen: \textit{Tier} [ti͡ɐ], \textit{Tür} [ty͡ɐ], \textit{rotem} [ʁoːtəm],\\
      \textit{Lob} [loːp], \textit{Bades} [baːdəs], \textit{Pfanne} [p͡fanə], \textit{Osten} [ʔɔstən]
      \vspace{\baselineskip}
    \item Warum gibt es genau die Basiszeichen im IPA, die es gibt? (a, ə, ɪ, ʔ, p, ʁ usw.)
      \begin{itemize}
        \item \alert{artikulatorische Untrennbarkeit}
        \item \alert{kein autonomes Verhalten potentieller Teile}
      \end{itemize}
      \vspace{\baselineskip}
    \item Sind p͡f und a͡ɔ usw.\ ein oder zwei Segmente? 
      \begin{itemize}
        \item artikulatorisch trennbar
        \item autonomes Verhalten?
        \item eigentlich eine phonologische Frage → Verteilungen
      \end{itemize}
  \end{itemize}
\end{frame}

\begin{frame}
  {Verteilungen: Beispiele}
  \pause
  \begin{exe}
    \ex
      \begin{xlist}
        \ex Tod [\alert{t}oːt], Kot [\alert{k}oːt]
        \pause
        \ex Schott [ʃɔ\alert{t}], Schock [ʃɔ\alert{k}]
      \end{xlist}
        \pause
        \ex Hang [ha\alert{ŋ}], *[\orongsch{ŋ}ah]
        \pause
    \ex
      \begin{xlist}
        \ex Sog [\alert{z}oːk], besingen [bə\alert{z}ɪŋən], *[\orongsch{s}oːk]
        \pause
        \ex fließ [fliː\alert{s}], Boss [bɔ\alert{s}], *[fliː\orongsch{z}]
        \pause
        \ex heißer [ha͡ɛ\alert{s}ɐ], heiser [ha͡ɛ\alert{z}ɐ], Base [baː\alert{z}ə], Basse [ba\alert{s}ə], *[ba\orongsch{z}ə]
      \end{xlist}
  \end{exe}
\end{frame}


\begin{frame}
  {Verteilung: Definition}
  \pause
  \Large
  \begin{block}{Verteilung}
    Die Verteilung eines Segments ist die Menge der Umgebungen, in denen es vorkommt.
  \end{block}
  \pause
  \Zeile
  \begin{block}{Kontrast}
    Zwei phonetisch unterschiedliche Segmente bzw.\ Merkmale stehen in einem phonologischen 
  Kontrast, wenn sie eine teilweise oder vollständig übereinstimmende Verteilung haben und dadurch einen lexikalischen bzw.\ grammatischen Unterschied markieren können.
  \end{block}
\end{frame}

\begin{frame}
  {Neutralisierung: Beispiele}
  \pause
  \begin{exe}
    \ex
    \begin{xlist}
      \ex{Weg [veːk], Weges [veːgəs]}
      \pause
      \ex{Bock [bɔk], Bockes [bɔkəs]}
      \pause
    \end{xlist}
    \ex
    \begin{xlist}
      \ex{Bad [baːt], Bades [baːdəs]}
      \pause
      \ex{Blatt [blat], Blattes [blatəs]}
      \pause
    \end{xlist}
    \ex
    \begin{xlist}
      \ex{Lob [loːp], Lobes [loːbəs]}
      \pause
      \ex{Depp [dɛp], Deppen [dɛpən]}
      \pause
    \end{xlist}
    \ex
    \begin{xlist}
      \ex aktiv [ʔaktiːf], aktive [ʔaktiːvə]
      \pause
      \ex tief [tiːf], tiefe [tiːfə]
      \pause
    \end{xlist}
    \ex
    \begin{xlist}
      \ex fies [f"|iːs], fiese [f"|iːzə]
      \pause
      \ex Bus [bʊs], Busse [bʊsə]
      \pause
    \end{xlist}
  \end{exe}
\end{frame}

\begin{frame}
  {Neutralisierung: Definition}
  \pause
  \Large
  \begin{block}{Neutralisierung}
    Eine Neutralisierung ist die Aufhebung eines phonologischen Kontrasts in einer bestimmten Position.    
  \end{block}
\end{frame}

\begin{frame}
  {Das Lexikon (Kapitel 2)}
  \pause
  \large Zum Verständnis der Phonologie ist der linguistische Begriff\\
  des Lexikons eine Grundvoraussetzung.\\
  \Large
  \Zeile
  \pause
  \begin{block}{Lexikon}
    Das \alert{Lexikon} ist die Menge aller Wörter einer Sprache, definiert durch die vollständige Angabe ihrer Merkmale und deren Werte.    
  \end{block}
  \pause
  \Zeile
  \large
  In der Phonologie ist das relevante Merkmal die \alert{Kette von Segmenten}, die ein Wort eindeutig definiert und von allen anderen Wörtern unterscheidbar macht.
\end{frame}

\begin{frame}
  {Muss man ʔ lexikalisch spezifizieren?}
  \pause
  \begin{itemize}[<+->]
    \item{[ʔan], [dan], [kan], [ʁan], [van], [man], [ban]}
    \item{[ʔoːnə], [boːnə], [loːnə], [t͡soːnə], [foːnə], [moːnə], [zoːnə]}
    \item{[ʔe͡ɐt], [ve͡ɐt], [le͡ɐt], [ke͡ɐt], [te͡ɐt], [ge͡ɐt], [he͡ɐt]}
  \end{itemize}
  \Zeile
  \pause
  \begin{itemize}[<+->]
    \item{\alert{[ʔ] wird immer am Silbenanfang eingesetzt,\\
      wenn dort lexialisch kein Konsonant zur Verfügung steht.}}
    \item{[ʔ] ist artikulatorisch und perzeptorisch wenig salient.}
    \item also: nicht lexikalisch, \alert{automatisch einsetzbar}
  \end{itemize}
\end{frame}

\begin{frame}
  {Endrand-Desonorisierung}
  \pause
  \begin{exe}
    \ex
    \begin{xlist}
      \ex{Weg [veː\alert{k}], Weges [veː\alert{g}əs]}
      \ex{Bock [bɔ\alert{k}], Bockes [bɔ\alert{k}əs]}
    \end{xlist}
    \ex
    \begin{xlist}
      \ex{Bad [baː\alert{t}], Bades [baː\alert{d}əs]}
      \ex{Blatt [bla\alert{t}], Blattes [bla\alert{t}əs]}
    \end{xlist}
    \ex
    \begin{xlist}
      \ex{Lob [loː\alert{p}], Lobes [loː\alert{b}əs]}
      \ex{Depp [dɛ\alert{p}], Deppen [dɛ\alert{p}ən]}
    \end{xlist}
    \ex
    \begin{xlist}
      \ex aktiv [ʔaktiː\alert{F}], aktive [ʔaktiː\alert{v}ə]
      \ex tief [tiː\alert{f}], tiefe [tiː\alert{f}ə]
    \end{xlist}
    \ex
    \begin{xlist}
      \ex fies [f"|iː\alert{s}], fiese [f"|iː\alert{z}ə]
      \ex Bus [bʊ\alert{s}], Busse [bʊ\alert{s}ə]
    \end{xlist}
  \end{exe}
  \pause
  \Zeile
  \begin{itemize}
    \item \alert{Aus welcher Form kann man die andere jeweils "`herleiten"'?}
  \end{itemize}
\end{frame}


\begin{frame}
  {Zugrundeliegende Form und Strukturbedingung}
  \pause
  \Large
  \begin{block}{Zugrundeliegende Form}    
    Die zugrundeliegende Form (eines Wortes) ist genau die Folge von Segmenten, die im Lexikon gespeichert wird, und auf die alle zugehörigen phonetischen Formen zurückgeführt werden können.
  \end{block}
  \pause
  \Zeile
  \begin{block}{Strukturbedingungen}
    Die Formen werden ggf. an die phonologischen Strukturbedingungen (die Regularitäten der phonologischen Grammatik) angepasst.    
  \end{block}
\end{frame}

\begin{frame}
  {Architektur der Grammatik und externer Systeme}
  \pause
  \centering
  \resizebox{0.9\textwidth}{!}{
    \begin{tabular}{ccc}
      \toprule
      \multicolumn{2}{c}{\textbf{Grammatik}} & \textbf{Externe Systeme} \\
      \midrule
      \textbf{Lexikon} & \textbf{Phonologie} & \textbf{Phonetik} \\
      \midrule
      /~/& $\Rightarrow$ & [~]\\
      zugrundeliegende Form & Anpassung an Strukturbedingungen & phonetische Realisierung \\
      \bottomrule
    \end{tabular}
  }
\end{frame}

\begin{frame}
  {Also für ʔ und Endrand-Desonorisierung}
  \pause
  \begin{itemize}[<+->]
    \item ʔ
      \begin{itemize}[<+->]
        \item /an/ $\Rightarrow$ [\alert{ʔ}an] 
        \item /oːnə/ $\Rightarrow$ [\alert{ʔ}oːnə]
        \item /e͡ɐt/ $\Rightarrow$ [\alert{ʔ}e͡ɐt]
      \end{itemize}
      \Zeile
    \item Endrand-Desonorisierung
      \begin{itemize}[<+->]
        \item /veː\gruen{g}/ $\Rightarrow$ [veː\alert{k}], /bɔ\gruen{k}/ $\Rightarrow$ [bɔ\alert{k}]
        \item /baː\gruen{d}/ $\Rightarrow$ [baː\alert{k}], /bla\gruen{t}/ $\Rightarrow$ [bla\alert{t}]
        \item /loː\gruen{b}/ $\Rightarrow$ [loː\alert{p}], /dɛ\gruen{p}/ $\Rightarrow$ [dɛ\alert{p}]
        \item /aktiː\gruen{v}/ $\Rightarrow$ [ʔaktiː\alert{f}], /tiː\gruen{f}/ $\Rightarrow$ [tiː\alert{f}]
        \item /fiː\gruen{z}/ $\Rightarrow$ [f"|iː\alert{s}], /bʊ\gruen{s}/ $\Rightarrow$ [bʊ\alert{s}]
      \end{itemize}
  \end{itemize}
\end{frame}


\begin{frame}
  {Endrand-Desonorisierung als Strukturbedingung}
  \pause
  \Large
  Alle \alert{Obstruenten} sind \alert{stimmlos} am \alert{Silbenende}.
\end{frame}


\begin{frame}
  {Verteilung von [ç] und [χ]}
  \pause
  \begin{exe}
    \ex
    \begin{xlist}
      \ex krieche, schlich, Bücher, Küche, Recht, Köche
      \pause
      \ex Tuch, Geruch, hoch, Koch, Schmach, Bach
    \end{xlist}
  \end{exe}
  \pause
  \Zeile
  \Large
  [ç] kann nicht nach nicht-vorderen Vokalen stehen.\\
  Zugrundeliegendes /ç/ wird daher\\
  nach zentralen und hinteren Vokalen\\
  weiter hinten artikuliert, nämlich als [χ].
\end{frame}

\begin{frame}
  {r-Vokalisierung}
  \pause
  \begin{exe}
    \ex
    \begin{xlist}
      \ex \textit{kleiner} [kla͡ɛ.nɐ], \textit{kleinere} [kla͡ɛ.nə.ʁə]
      \pause
      \ex \textit{Bär} [bɛ͡ɐ], \textit{Bären} [bɛː.ʁən]
      \pause
      \ex \textit{knarr} [kna͡ə], \textit{knarre} [kna.ʁə]
    \end{xlist}
  \end{exe}
  \pause
  \Zeile
  \Large
  Zugrundeliegendes /ʁ/ kann nicht am Silbenende\\
  stehen. Es wird in dieser Position als\\
  Schwa-Segment im sekundären Diphthong\\
  realisiert. Nach gespanntem Vokal folgt [ɐ],\\
  nach ungespanntem folgt [ə]. Schwa und /ʁ/\\
  werden zusammen durch [ɐ] substituiert.\\[0.5\baselineskip]
  \pause
  \alert{Gespannt?}
\end{frame}


\begin{frame}[fragile]
  {Erinnerung an die Vokale des Deutschen}
  \begin{center}
  \resizebox{0.5\textwidth}{!}{
  \begin{tikzpicture}[scale=2.5,baseline=default]
    \large
    \tikzset{
      vowel/.style={fill=white, anchor=mid, text depth=0ex, text height=1ex},
      dot/.style={circle,fill=black,minimum size=0.4ex,inner sep=0pt,outer sep=-1pt},
    }

    \coordinate (hf) at (0,2); % high front
    \coordinate (hb) at (2,2); % high back
    \coordinate (lf) at (1,0); % low front
    \coordinate (lb) at (2,0); % low back
    \def\V(#1,#2){barycentric cs:hf={(3-#1)*(2-#2)},hb={(3-#1)*#2},lf={#1*(2-#2)},lb={#1*#2}}

    % Chart key (vorne -- hinten).
    \draw [{Latex[round]}-] (\V (-.25,0))   -- (\V (-.25,.5)) node [above left] {\footnotesize vorne};
    \draw [-{Latex[round]}] (\V (-.25,1.5)) -- (\V (-.25,2))  node [above left] {\footnotesize hinten};
    \path (\V (-.25,1)) node[above] {\footnotesize zentral};

    % Chart key (hoch--tief).
    \draw [{Latex[round]}-] (\V (0,-.25)) -- +(270:.5cm)  node [above right,rotate=90] (vokaltrapez1) {\footnotesize hoch};
    \draw [{Latex[round]}-] (\V (3,-2.5)) -- +(270:-.5cm) node [above left,rotate=90] (vokaltrapez2) {\footnotesize tief};
    \path (\V (1.5,-1)) node[above,rotate=90] {\footnotesize mittel};

    % Grid. 
    \draw [gray, thick] (\V(0,0)) -- (\V(0,2));
    \draw [gray, thick] (\V(1,0)) -- (\V(1,2));
    \draw [gray, thick] (\V(2,0)) -- (\V(2,2));
    \draw [gray, thick] (\V(3,0)) -- (\V(3,2));
    \draw [gray, thick] (\V(0,0)) -- (\V(3,0));
    \draw [gray, thick] (\V(0,1)) -- (\V(3,1));
    \draw [gray, thick] (\V(0,2)) -- (\V(3,2));

    % Unrounded-rounded pairs.
    \path (\V(0,0))     node[vowel, left]     {i} node[vowel, right] (y) {y} node[dot] {};
    \path (\V(0.5,0.5)) node[vowel, left]     {ɪ} node[vowel, right] (Y) {ʏ} node[dot] {};
    \path (\V(1,0))     node[vowel, left]     {e} node[vowel, right] (e) {ø} node[dot] {};
    \path (\V(2,0))     node[vowel, left] (E) {ɛ} node[vowel, right] (ee) {œ} node[dot] {};

    % Unpaired symbols.
    \path (\V(1.5,1))    node [vowel] (schwa)  {ə};
    \path (\V(2.5,1))    node [vowel] (schwaa) {ɐ};
    \path (\V(3,1))      node [vowel] (a)      {a};
    \path (\V (2,2))     node [vowel] (oo)     {ɔ};
    \path (\V (1,2))     node [vowel] (o)      {o};
    \path (\V (0,2))     node [vowel] (u)      {u};
    \path (\V (0.5,1.5)) node [vowel] (uu)     {ʊ};

  \end{tikzpicture}
  }
  \end{center}
\end{frame}


\begin{frame}
  {Länge und Betonung und Vokalqualität im Systemkern}
  \pause
  \centering
  \begin{tabular}{cllp{0.25cm}cll}
    \toprule
    \textbf{gespannt} & \textbf{Beispiel} & \textbf{IPA} & & \textbf{ungespannt} & \textbf{Beispiel} & \textbf{IPA} \\
    \midrule
    i  & \textit{bieten} & biːtən && ɪ & \textit{bitten}  & bɪtən   \\
    y  & \textit{fühlt}  & fyːlt  && ʏ & \textit{füllt}   & fʏlt    \\
    u  & \textit{Mus}    & muːs   && ʊ & \textit{muss}    & mʊs     \\
    e  & \textit{Kehle}  & keːlə  && ɛ & \textit{Kelle}   & kɛlə    \\
    ɛ  & \textit{stähle} & ʃtɛːlə && ɛ & \textit{Ställe}  & ʃtɛlə   \\
    ø  & \textit{Höhle}  & høːlə  && œ & \textit{Hölle}   & hœlə \\
    o  & \textit{Ofen}   & ʔoːfən && ɔ & \textit{offen}   & ʔɔfən   \\
    a  & \textit{Wahn}   & vaːn   && a & \textit{wann}    & van     \\
    \bottomrule
  \end{tabular}\\
  \pause
  \Zeile
  \begin{itemize}[<+->]
    \item Laut\rot{e}, b\rot{e}schreib\rot{e}n, \dots
    \item L\rot{i}thografie, H\rot{y}draulik, B\rot{u}tan, Ph\rot{e}nol, \rot{Ö}nologie, Mes\rot{o}zoon, \dots
  \end{itemize}
\end{frame}

\begin{frame}
  {Gespanntheit im Kernwortschatz}
  \pause
  \Large
  \rot{Im Kernwortschatz sind gespannte Vokale immer\\
  betont und lang.} Zu jedem gespannten Vokal gibt es\\
  einen entsprechenden ungespannten Vokal.\\
  Der ungespannte ist betont oder unbetont,\\
  aber immer kurz.\\
  \Zeile
  \pause
  Die Länge muss also nicht markiert werden, sondern folgt\\
  aus Betonung und Gespanntheit.
\end{frame}

\begin{frame}[fragile]
  {Gespanntheit}
  \pause
  \begin{center}
    \resizebox{0.5\textwidth}{!}{
      \begin{tikzpicture}[scale=3.5,baseline=default]
        \large
        \tikzset{
        vowel/.style={fill=white, anchor=mid, text depth=0ex, text height=1ex},
        vowelgespannt/.style={circle,fill=gray!30, anchor=mid, text depth=0ex, text height=1ex,minimum size=4ex},
        dot/.style={circle,fill=black,minimum size=0.4ex,inner sep=0pt,outer sep=-1pt},
        }

        \coordinate (hf) at (0,2); % high front
        \coordinate (hb) at (2,2); % high back
        \coordinate (lf) at (1,0); % low front
        \coordinate (lb) at (2,0); % low back
        \def\V(#1,#2){barycentric cs:hf={(3-#1)*(2-#2)},hb={(3-#1)*#2},lf={#1*(2-#2)},lb={#1*#2}}

        % Chart key (vorne -- hinten).
        \draw [{Latex[round]}-] (\V (-.25,0)) -- (\V (-.25,.5))  node [above left] {\footnotesize vorne};
        \draw [-{Latex[round]}] (\V (-.25,1.5)) -- (\V (-.25,2)) node [above left] {\footnotesize hinten};
        \path (\V (-.25,1)) node[above] {\footnotesize zentral};

        % Chart key (hoch--tief).
        \draw [{Latex[round]}-] (\V (0,-.25)) -- +(270:.5cm)  node [above right,rotate=90] (vokaltrapez1) {\footnotesize hoch};
        \draw [{Latex[round]}-] (\V (3,-2.5)) -- +(270:-.5cm) node [above left,rotate=90] (vokaltrapez2) {\footnotesize tief};
        \path (\V (1.5,-1)) node[above,rotate=90] {\footnotesize mittel};

        % Grid.
        \draw [gray,thick] (\V(0,0)) -- (\V(0,2));
        \draw [gray,thick] (\V(3,0)) -- (\V(3,2));
        \draw [gray,thick] (\V(0,0)) -- (\V(3,0));
        \draw [gray,thick] (\V(0,2)) -- (\V(3,2));

        \path (\V(0,0))      node[vowelgespannt] (i)   {i};
        \path (\V(0.25,0))   node[vowelgespannt] (y)   {y};
        \path (\V(0.4,0.5))  node[vowel]         (ii)  {ɪ};
        \path (\V(0.65,0.5)) node[vowel]         (yy)  {ʏ};
        \path (\V(1,0))      node[vowelgespannt] (e)   {e};
        \path (\V(1.25,0))   node[vowelgespannt] (oe)  {ø};
        \path (\V(2,0))      node[vowelgespannt] (ee)  {ɛ};
        \path (\V(1.4,0.7))  node[vowel]         (eee) {ɛ̆};
        \path (\V(1.65,0.7)) node[vowel]         (oee) {œ};
        \path (\V(3,1))      node[vowelgespannt] (a)   {a};
        \path (\V(2.5,1))    node[vowel]         (aa)  {ă};
        \path (\V (1,2))     node[vowelgespannt] (o)   {o};
        \path (\V (1.5,1.4)) node[vowel]         (oo)  {ɔ};
        \path (\V (0,2))     node[vowelgespannt] (u)   {u};
        \path (\V (0.5,1.5)) node[vowel]         (uu)  {ʊ};

        \draw (i)  -- (ii);
        \draw (y)  -- (yy);
        \draw (e)  -- (eee);
        \draw (oe) -- (oee);
        \draw (ee) -- (eee);
        \draw (a)  -- (aa);
        \draw (o)  -- (oo);
        \draw (u)  -- (uu);
      \end{tikzpicture}
    }
  \end{center}
\end{frame}


\begin{frame}
  {Und Schwa?}
  \pause
  Warum kommt Schwa (also [ə] und [ɐ]) im System der gespannten\\
  und ungespannten Vokale nicht vor?\\
  \pause
  \Zeile
  \Zeile
  \centering
  \Large
  \alert{Schwa ist nicht betonbar!}
\end{frame}



\begin{frame}
  {Und der erweiterte Wortschatz?}
  \resizebox{0.9\textwidth}{!}{
  \begin{minipage}{\textwidth}
  \begin{exe}
    \ex\label{ex:gespanntheitbetonungundlaenge021}
    \begin{xlist}
      \ex{\label{ex:gespanntheitbetonungundlaenge022} \textit{Idee} [ʔ\rot{i}deː]\\
      \textit{Initiative} [ʔ\rot{i}n\rot{i}t͡sʝatiːvə]\\
        \textit{inspirieren} [ʔɪnsp\rot{i}ʁiːʁən] }
      \ex{\label{ex:gespanntheitbetonungundlaenge023} \textit{Methyl} [m\rot{e}tyːl]\\
        \textit{Québec} [k\rot{e}bɛk]\\
        \textit{integriert} [ʔɪnt\rot{e}gʁi͡ɐt]\\
        \textit{debattieren} [d\rot{e}batiːʁən] }
      \ex{\label{ex:gespanntheitbetonungundlaenge024} \textit{Utopie} [ʔ\rot{u}topiː]\\
        \textit{Uran} [ʔ\rot{u}ʁaːn] }
      \ex{\label{ex:gespanntheitbetonungundlaenge025} \textit{Motiv} [m\rot{o}tiːf]\\
        \textit{politisch} [p\rot{o}liːtɪʃ]\\
        \textit{Phonologie} [f\rot{o}n\rot{o}l\rot{o}giː] }
      \ex{\label{ex:gespanntheitbetonungundlaenge026} \textit{Ökonomie} [ʔ\rot{ø}konomiː]\\
        \textit{manövrieren} [man\rot{ø}vʁiːʁən] }
      \ex{\label{ex:gespanntheitbetonungundlaenge027} \textit{Büro} [b\rot{y}ʁoː]\\
      \textit{Cuvée} [k\rot{y}veː] }
    \end{xlist}
  \end{exe}
  \end{minipage}
  }
\end{frame}

\begin{frame}
  {Gespanntheit im erweiterten Wortschatz}
  \pause
  \Large
  \rot{Im erweiterten Wortschatz sind gespannte Vokale\\
  lang, wenn sie betont sind, und kurz, wenn sie \\
  unbetont sind.} Auch im erweiterten Wortschatz\\
  gibt es keine ungespannten langen Vokale.\\
\end{frame}

\begin{frame}
  {Zugrundeliegende Formen ohne Länge}
  \begin{exe}
    \ex\label{ex:gespanntheitbetonungundlaenge028} \begin{xlist}
      \ex /v\rot{e}g/ $\Rightarrow$ [v\rot{e}ːk]
      \ex /h\rot{ø}lə/ $\Rightarrow$ [h\rot{ø}ːlə]
      \ex /\rot{o}fən/ $\Rightarrow$ [ʔ\rot{o}ːfən]
    \end{xlist}
  \end{exe}
\end{frame}

\ifdefined\TITLE
  \section{Nächste Woche | Überblick}

  \begin{frame}
    {Der ungefähre Semesterplan}
    \begin{enumerate}[<+->]
      \item Graphematik und Schreibprinzipien
      \item Wiederholung -- Phonetik
      \item Wiederholung -- Phonologie
      \item \alert{Phonographisches Schreibprinzip -- Konsonanten}
      \item Phonographisches Schreibprinzip -- Vokale
      \item Silben und Dehnungsschreibungen
      \item Eszett, Dehnung und Konstanz
      \item Spatien und Majuskeln
      \item Komma
      \item Punkt und sonstige Interpunktion
    \end{enumerate}
  \end{frame}
\fi


\ifdefined\TITLE
  \section{Nächste Woche | Überblick}

  \begin{frame}
    {Semesterplan}
    \begin{enumerate}
      \item Graphematik und Schreibprinzipien
      \item \alert{Wiederholung -- Phonetik}
      \item Wiederholung -- Phonologie
      \item Phonographisches Schreibprinzip -- Konsonanten
      \item Phonographisches Schreibprinzip -- Vokale
      \item Silben und Dehnungsschreibungen
      \item Eszett, Dehnung und Konstanz
      \item Spatien und Majuskeln
      \item Komma
      \item Punkt und sonstige Interpunktion
    \end{enumerate}
  \end{frame}
\fi
