
\section{Organisation}

\begin{frame}
  {Ablauf und Inhalte der Vorlesung}
  \begin{itemize}
    \item 10 Sitzungen über Phonetik, Phonologie und Graphematik des Deutschen 
    \item Größere Teile des Inhalts in meiner \alert{\textit{Einführung in\\
      die grammatische Beschreibung des Deutschen}} \grau{\citep{Schaefer2018b}}
    \item \url{http://langsci-press.org/catalog/book/224} (\alert{open access})
      \vspace{\baselineskip}
    \item Bei Amazon für 20€\\
      \url{https://www.amazon.de/dp/3961101183/}
  \end{itemize}
\end{frame}

\begin{frame}
  {Der Plan für heute}
  \pause
  \begin{itemize}
    \item Graphematik als Teil der Grammatik
    \item Schreibprinzipien
  \end{itemize}
\end{frame}


\section[Graphematik]{Graphematik als Teil der Grammatik}

\begin{frame}
  {Schrift und Schreibung}
  \onslide<+->
  \onslide<+->
  \alert{Schrift}\\
  \Halbzeile
  \begin{itemize}[<+->]
    \item das Inventar von Schriftzeichen
    \item ihre Funktion und Relevanz als einzelnes Zeichen im System
  \end{itemize}
  \onslide<+->
  \Zeile
  \alert{Schreibung}\\
  \Halbzeile
  \begin{itemize}[<+->]
    \item der Aufbau größerer geschriebener Strukturen
    \item Wörter
    \item Wortgruppen
    \item Sätze
    \item einschließlich Interpunktion
  \end{itemize}
\end{frame}

\begin{frame}
  {Graphematik | Kodierung von Grammatik in Schriftartefakten}
  \alert{Phonologische Prinzipien}\\
  \Halbzeile
  \begin{itemize}[<+->]
    \item /r/ → <r> | \textit{Rat} [ʁaːt], \textit{Bar} [ba͡ə]
    \item /ṭ/ → <tt> | \textit{Matte} [maṭə]
  \end{itemize}
  \Zeile
  \alert{Morphologische\slash Lexikologische Prinzipien}\\
  \Halbzeile
  \begin{itemize}[<+->]
    \item Stammkonstanz | <Tri\rot{tt}> [tʁɪt] ← <Tri\gruen{tt}e> [tʁɪṭə] 
  \end{itemize}
  \Zeile
  \alert{Syntax}
  \Halbzeile
  \begin{itemize}[<+->]
    \item syntaktisches Wort → < > | \textit{Haustür}, \textit{die Tür}
    \item Nebensatzeinbettung → <,> | \textit{Ich weiß, dass es regnet.}
  \end{itemize}
\end{frame}


\begin{frame}
  {Graphematik in ihrem Element | Was ist hier falsch?}
  \onslide<+->
  \onslide<+-> 
  \begin{exe}
    \ex\label{ex:graphematikalsteildergrammatik001}
    \begin{xlist}
      \ex[*]{\label{ex:graphematikalsteildergrammatik002} Fine findet, \rot{das} die Schuhe gut aussehen.}
      \onslide<+->
      \ex[*]{\label{ex:graphematikalsteildergrammatik003} Wenn ich Geld hätte, \rot{nehme} ich den Kopfhörer mit.}
      \onslide<+->
      \ex[*]{\label{ex:graphematikalsteildergrammatik004} Um voranzukommen, nimmt Fine an der Fortbildung \rot{Teil}.}
      \onslide<+->
      \ex[*]{\label{ex:graphematikalsteildergrammatik005} \rot{Zurückbleibt} der Schreibtisch nur, wenn der LKW randvoll ist.}
    \end{xlist}
  \end{exe}
  \begin{itemize}[<+->]
    \item falsche lexikalische Schreibung → Wort existiert,\\
      \alert{hier falsche Wortklasse}
    \item falsche Segmentschreibung → Form möglich, \alert{hier falsche Flexionsform}
    \item falsche Wort(klassen)schreibung → Wort existiert,\\
      \alert{hier falscher morphosyntaktischer Status}
    \item falsche Wortschreibung (Spatium) → \textit{zurückbleibt} anderswo möglich\\
      \alert{hier durch Bewegungssyntax ausgeschlossen}
  \end{itemize}
\end{frame}

\begin{frame}
  {Das Primat der gesprochenen Sprache \ldots\ Eine prima Ente!}
  \orongsch{Schrift ist nicht natürlich, sondern ausgedacht!}\\
  \Viertelzeile
  \begin{itemize}[<+->]
    \item Siehe Geschichte der Schrift.
  \end{itemize}
  \Halbzeile
  \orongsch{Gesprochene Sprache ist im Erwerb primär!}
  \Viertelzeile
  \begin{itemize}[<+->]
    \item Wieso sollte das heißen, Schrift wäre irrelevant?
    \item Einfluss der Entwicklung der motorischen Fähigkeiten?\\
      \grau{\footnotesize Stift mit drei Fingern halten: ab 3 Jahre ; Kreis malen: ab 3,5 Jahre}
  \end{itemize}
  \Halbzeile
  \orongsch{Es gibt Sprachen ohne Schrift!}\\
  \Viertelzeile
  \begin{itemize}[<+->]
    \item Es gibt Atome ohne Elektronen, vergesst Atome mit Elektronen!
  \end{itemize}
  \Halbzeile
  \orongsch{Schrift wird nicht spontan produziert!}
  \Viertelzeile
  \begin{itemize}[<+->]
    \item Bildungssprache ist immer überformt, auch im gesprochenen Modus.
    \item \gruen{Wie wir sehen werden, wird oft sehr spontan geschrieben.}
  \end{itemize}
\end{frame}



\section[Prinzipien]{Schreibprinzipien, illustriert}

\begin{frame}
  {Schreibprinzipien -- oder auch nicht}
  \centering 
  \includegraphics[width=0.9\textwidth]{graphics/wii}\\
  \Halbzeile
  \grau{\tiny Hannah aus Berlin mit 6 Jahren}
\end{frame}

\newcommand{\graphem}[1]{\ensuremath{\langle}#1\ensuremath{\rangle}}

\begin{frame}
  {Von welchen Schreibprinzipien weicht Hannah ab?}
  \centering 
  \includegraphics[width=0.2\textwidth]{graphics/wii}\\
  \Halbzeile
  \raggedright
  \begin{itemize}[<+->]
    \item Prinzipien der \alert{Majuskelschreibung} nicht gelernt
    \item Prinzip der \alert{Spatienschreibung} nicht gelernt
    \item \alert{\graphem{WAN}} | \alert{keine} Prinzipverletzung
    \item \alert{\graphem{DAF}} | \alert{phonetische} Abweichung vom Standard
    \item \alert{\graphem{ich}} | einwandfrei
    \item \alert{\graphem{Wii}} | \graphem{ii}-Dehnungsschreibung atypisch, \alert{Produktname}
    \item \graphem{\alert{sch}BiLN} | \alert{Abweichung von Prinzip} (Segmentschreibung) nicht gelernt
    \item \graphem{sch\alert{B}iLN} | \alert{phonetisch-phonologisches} "`Problem"'
    \item \graphem{schB\alert{i}LN} | \graphem{ie}-typische Dehnungsschreibung nicht gelernt
    \item \graphem{schBiL\alert{N}} | \alert{phonetische} Abweichung vom Standard
  \end{itemize}
\end{frame}

\begin{frame}
  {Warum kann die Schülerin nichts dafür?}
  \begin{itemize}[<+->]
    \item \alert{Hinhörschreibung} | Wir schreiben nicht, wie wir sprechen!\\
      "`Hinhören"' kann Hannah sehr gut.
      \Zeile
    \item \alert{Ausprobierschreibung} | \alert{Abweichungen von den Prinzipien}\\
      werden nicht beherrscht. Das ist das Ergebnis des Ausprobierens.
    \item Was wir uns selber erarbeiten (= ausprobieren), merken wir uns besonders gut.
      \Zeile
    \item Harte Prinzipien wurden nicht unterrichtet (Spatien, Majuskeln).
  \end{itemize}
\end{frame}

\ifdefined\TITLE
  \section{Nächste Woche | Überblick}

  \begin{frame}
    {Semesterplan}
    \begin{enumerate}
      \item Graphematik und Schreibprinzipien
      \item \alert{Wiederholung -- Phonetik}
      \item Wiederholung -- Phonologie
      \item Phonographisches Schreibprinzip -- Konsonanten
      \item Phonographisches Schreibprinzip -- Vokale
      \item Silben und Dehnungsschreibungen
      \item Eszett, Dehnung und Konstanz
      \item Spatien und Majuskeln
      \item Komma
      \item Punkt und sonstige Interpunktion
    \end{enumerate}
  \end{frame}
\fi
