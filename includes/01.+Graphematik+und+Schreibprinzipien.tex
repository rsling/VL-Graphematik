
\section{Organisation}

\begin{frame}
  {Roland Schäfer}
  \onslide<+->
  \begin{itemize}[<+->]
    \item seit WS 2022\slash 2023 Professur für Grammatik und Lexikon
    \item 2020--2022 Forschungsstelle an der HU Berlin
    \item 2018 habilitiert an der HU Berlin\\
      (Germanistische Linguistik und allgemeine Sprachwissenschaft)
    \item 2007--2022 Mitarbeiter an der FU Berlin
    \item 2008 promoviert an der Uni Göttingen (Englische Syntax)
    \item 2002--2007 Mitarbeiter in der Sprachwissenschaft in Göttingen
    \item Studium in Marburg (Sprachwissenschaft, Japanologie)
  \end{itemize}
  \Zeile
  \onslide<+->
  Bitte nennen Sie mich nicht Professor\ldots\ \onslide<+-> Wenn Sie es tun, dann bitte richtig:\\
  \url{https://rolandschaefer.net/regeln-fur-den-mailverkehr/}
\end{frame}

\begin{frame}
  {Forschung}
  \onslide<+->
  \onslide<+->
  Linguistik (des Deutschen)\\
  \Halbzeile
  \begin{itemize}[<+->]
    \item kognitiv fundierte Grammatik
    \item Morphosyntax und Graphematik
    \item grammatische Variation ("`Zweifelsfälle"')
    \item individuelle Variation
    \item Registervariation
    \item Epistemologie
  \end{itemize}
  \Zeile
  \onslide<+->
  Methoden\\
  \Halbzeile
  \begin{itemize}[<+->]
    \item Korpuserstellung und -analyse
    \item verhaltensbasierte Experimente
    \item Fragen der statistischen Inferenz
  \end{itemize}
\end{frame}

\begin{frame}
  {Ablauf und Inhalte der Vorlesung}
  \begin{itemize}
    \item 13 Sitzungen über Graphematik des Deutschen
    \item Größere Teile des Inhalts in meiner \alert{\textit{Einführung in\\
      die grammatische Beschreibung des Deutschen}} \grau{\citep{Schaefer2018b}}
    \item \url{http://langsci-press.org/catalog/book/224} (\alert{open access})
      \vspace{\baselineskip}
    \item Bei Amazon für 20€\\
      \url{https://www.amazon.de/dp/3961101183/}
  \end{itemize}
\end{frame}

\begin{frame}
  {Fragen und Interaktion}
  \begin{itemize}
    \item Interaktion in einer VL ist immer schwierig!\\
      Ich versuche es ggf.\ trotzdem.
      \Zeile
    \item Wenn Sie Fragen zum Stoff oder zum Buch haben:
      \texttt{roland.schaefer@uni-jena.de}
      \Zeile
    \item Mein Youtube-Kanal (demnächst wieder lebendig):\\
      \url{https://www.youtube.com/channel/UCc0SUpRSVvU2jJxx4rRBdsg}
  \end{itemize}
\end{frame}

\begin{frame}
  {Der Plan für heute}
  \pause
  \begin{itemize}
    \item Graphematik als Teil der Grammatik
    \item Schreibprinzipien
    \Zeile
    \item EGBD3: Kapitel 1
  \end{itemize}
\end{frame}


\section[Graphematik]{Graphematik als Teil der Grammatik}

\begin{frame}
  {Schrift und Schreibung}
  \onslide<+->
  \onslide<+->
  \alert{Schrift}\\
  \Halbzeile
  \begin{itemize}[<+->]
    \item das Inventar von Schriftzeichen
    \item ihre Funktion und Relevanz als einzelnes Zeichen im System
  \end{itemize}
  \onslide<+->
  \Zeile
  \alert{Schreibung}\\
  \Halbzeile
  \begin{itemize}[<+->]
    \item der Aufbau größerer geschriebener Strukturen
    \item Wörter
    \item Wortgruppen
    \item Sätze
    \item einschließlich Interpunktion
  \end{itemize}
\end{frame}

\begin{frame}
  {Graphematik in ihrem Element | Was ist hier falsch?}
  \onslide<+->
  \onslide<+-> 
  \begin{exe}
    \ex\label{ex:graphematikalsteildergrammatik001}
    \begin{xlist}
      \ex[*]{\label{ex:graphematikalsteildergrammatik002} Fine findet, \rot{das} die Schuhe gut aussehen.}
      \onslide<+->
      \ex[*]{\label{ex:graphematikalsteildergrammatik003} Wenn ich Geld hätte, \rot{nehme} ich den Kopfhörer mit.}
      \onslide<+->
      \ex[*]{\label{ex:graphematikalsteildergrammatik004} Um voranzukommen, nimmt Fine an der Fortbildung \rot{Teil}.}
      \onslide<+->
      \ex[*]{\label{ex:graphematikalsteildergrammatik005} \rot{Zurückbleibt} der Schreibtisch nur, wenn der LKW randvoll ist.}
    \end{xlist}
  \end{exe}
  \begin{itemize}[<+->]
    \item falsche lexikalische Schreibung → Wort existiert,\\
      \alert{hier falsche Wortklasse}
    \item falsche Segmentschreibung → Form möglich, \alert{hier falsche Flexionsform}
    \item falsche Wort(klassen)schreibung → Wort existiert,\\
      \alert{hier falscher morphosyntaktischer Status}
    \item falsche Wortschreibung (Spatium) → \textit{zurückbleibt} anderswo möglich\\
      \alert{hier durch Bewegungssyntax ausgeschlossen}
  \end{itemize}
\end{frame}

\begin{frame}
  {Einordnung und andere Meinungen I}
  \pause
  \begin{itemize}[<+->]
    \item Graphematik als eins der \alert{Kodierungssysteme der Grammatik}
    \item Relevanzunterschied zu Phonetik (= anderes Medium)? --- \alert{Keiner!}
    \item \alert{Natürlich gehört die Graphematik zur Grammatik\slash Linguistik.}
    \Zeile
    \item \rot{"`Aber viele Sprachen haben keine Schriftsysteme!"'}
      \begin{itemize}[<+->]
        \item \alert{Ja und? Viele haben eins, \zB das Deutsche.}
      \end{itemize}
      \Viertelzeile
    \item \rot{"`Aber es gibt Sprachen ohne Schrift und keine Schrift ohne Sprache!"'}
      \begin{itemize}[<+->]
        \item \alert{Ja und? Im Gegenteil: In \textit{Kulturen}, die Jahrhunderte oder -tausende lang\\
        verschriften, gibt es erhebliche Rückkopplungen zwischen\\
        Gesprochenem und Geschriebenem, \zB\ im Deutschen.}
      \end{itemize}
      \Viertelzeile
    \item \rot{"`Aber die Schrift haben sich Leute ausgedacht!"'}\\
      (soll heißen: Die Schreibung hat sich nicht natürlich entwickelt.)
      \begin{itemize}[<+->]
        \item \alert{Ach? Schonmal die Entwicklung der deutschen Schreibung angesehen?}
      \end{itemize}
  \end{itemize}
\end{frame}

\begin{frame}
  {Einordnung und andere Meinungen II}
  \pause
  \begin{itemize}[<+->]
    \item \rot{"`Aber die Schriftsprache ist nicht spontan, daher uninteressant\\
      für Linguistik (= Erforschung unbewusster kognitiver Vorgänge)!"'}
      \begin{itemize}[<+->]
        \item \alert{Ach? Sagen Linguisten, die glauben, dass sie selber (oder andere)\\
          durch Introspektion an ihre interne Grammatik rankommen!}
        \item Bildungssprache tendiert generell zur reflektierten \alert{Überformung},\\
          das Medium spielt dafür nur tendentiell eine Rolle.
      \end{itemize}
      \Viertelzeile
    \item \rot{"`Aber Kinder lernen zuerst Sprechen, ohne Schrift!"'}
      \begin{itemize}[<+->]
        \item \alert{Ja und? Wir beschreiben beide Kodierungssysteme ja auch getrennt.\\
          Niemand sagt, dass das dasselbe ist.}
        \item Das akustische Medium hat meist aus praktischen Gründen Vorrang\\
          (aber vgl.\ \zB gehörlose Kinder).
      \end{itemize}
  \end{itemize}
\end{frame}

\begin{frame}
  {Einordnung und andere Meinungen III}
  \pause
  \begin{itemize}[<+->]
    \item \rot{"`Aber aus diesen } (falschen) \rot{Gründen, hält die gesprochene Sprache\\
      in der Linguistik traditionell das Primat über die geschriebene!"'}
      \begin{itemize}[<+->]
        \item \alert{Blanker Unsinn. Die meisten Linguisten, die sowas behaupten,\\
          haben vor allem keine Ahnung von gesprochener Sprache.}
        \item \grau{Vgl.\ \citet{Schwitalla2011} zur Einführung in gesprochene Sprache.}
      \end{itemize}
  \end{itemize}
\end{frame}



\section[Prinzipien]{Schreibprinzipien, illustriert}

\begin{frame}
  {Schreibprinzipien -- oder auch nicht}
  \centering 
  \includegraphics[width=0.9\textwidth]{graphics/wii}\\
  \Halbzeile
  \grau{\tiny Hannah aus Berlin mit 6 Jahren}
\end{frame}

\newcommand{\graphem}[1]{\ensuremath{\langle}#1\ensuremath{\rangle}}

\begin{frame}
  {Von welchen Schreibprinzipien weicht Hannah ab?}
  \centering 
  \includegraphics[width=0.2\textwidth]{graphics/wii}\\
  \Halbzeile
  \raggedright
  \begin{itemize}[<+->]
    \item Prinzipien der \alert{Majuskelschreibung} nicht gelernt
    \item Prinzip der \alert{Spatienschreibung} nicht gelernt
    \item \alert{\graphem{WAN}} | \alert{keine} Prinzipverletzung
    \item \alert{\graphem{DAF}} | \alert{phonetische} Abweichung vom Standard
    \item \alert{\graphem{ich}} | einwandfrei
    \item \alert{\graphem{Wii}} | \graphem{ii}-Dehnungsschreibung atypisch, \alert{Produktname}
    \item \graphem{\alert{sch}BiLN} | \alert{Abweichung von Prinzip} (Segmentschreibung) nicht gelernt
    \item \graphem{sch\alert{B}iLN} | \alert{phonetisch-phonologisches} "`Problem"'
    \item \graphem{schB\alert{i}LN} | \graphem{ie}-typische Dehnungsschreibung nicht gelernt
    \item \graphem{schBiL\alert{N}} | \alert{phonetische} Abweichung vom Standard
  \end{itemize}
\end{frame}

\begin{frame}
  {Warum kann die Schülerin nichts dafür?}
  \begin{itemize}[<+->]
    \item \alert{Hinhörschreibung} | Wir schreiben nicht, wie wir sprechen!\\
      "`Hinhören"' kann Hannah sehr gut.
      \Zeile
    \item \alert{Ausprobierschreibung} | \alert{Abweichungen von den Prinzipien}\\
      werden nicht beherrscht. Das ist das Ergebnis des Ausprobierens.
    \item Was wir uns selber erarbeiten (= ausprobieren),\\
      merken wir uns besonders gut.
      \Zeile
    \item Harte Prinzipien wurden nicht unterrichtet (Spatien, Majuskeln).
  \end{itemize}
\end{frame}

\ifdefined\TITLE
  \section{Nächste Woche | Überblick}

  \begin{frame}
    {Der ungefähre Semesterplan}
    \begin{enumerate}[<+->]
      \item Graphematik und Schreibprinzipien
      \item \alert{Wiederholung -- Phonetik}
      \item Wiederholung -- Phonologie
      \item Phonographisches Schreibprinzip -- Konsonanten
      \item Phonographisches Schreibprinzip -- Vokale
      \item Silben und Dehnungsschreibungen
      \item Eszett, Dehnung und Konstanz
      \item Spatien und Majuskeln
      \item Komma
      \item Punkt und sonstige Interpunktion
    \end{enumerate}
  \end{frame}
\fi
