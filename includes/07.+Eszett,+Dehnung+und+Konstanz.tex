\section{Übersicht}

\begin{frame}
  {Übersicht}
  \onslide<+->
  \begin{itemize}[<+->]
    \item Wozu brauchen wir das \alert{Eszett}?
      \Zeile
    \item \alert{Konstanzprinzip} | Stämme möglichst konstant schreiben
      \Zeile
    \item Fazit | \rot{Kann die Dehnungsschreibung weg?}
  \end{itemize}
\end{frame}


\section{Eszett}

\newcommand{\phopro}{\ensuremath{\Rightarrow}}

\begin{frame}
  {Analyse des Eszett}
  \pause
  \begin{itemize}[<+->]
    \item \alert{Alle Positionen bis auf die \textit{ß}-Umgebung sind herleitbar:}
      \begin{itemize}[<+->]
        \item Wortanlaut (\textit{Sog} [zoːk]): zugrundeliegendes /z/ bleibt [z]
        \item Wortauslaut (\textit{Mus} [muːs]): zugrundeliegendes /z/ würde sowieso [s]\\
          wegen Endrand-Desonorisierung
        \item Wortinneren nach ungespanntem Vokal (\textit{Masse} [maṣə]): \alert{Silbengelenk}\\
          immer stimmlos wegen Endranddesonorisierung (/măzə/ undenkbar)
      \end{itemize}
      \Halbzeile
    \item \alert{Bis hierhin brauchen wir noch kein zugrundeliegendes /s/!}
      \Halbzeile
    \item zugrundeliegendes /s/ \rot{nur für das Wortinnere nach gespanntem Vokal}\\
      \textit{Straße} [ʃtʁaːsə] gegenüber \textit{Hase} [haːzə]
    \item \alert{Und wenn statt /s/ einfach /zz/ zugrundeliegt?}
    \item \alert{Und wenn /zz/ mit \textit{ß} geschrieben wird?}
    \item also: \textit{Bußen} als /buzzən/ \phopro [buːssən]
  \end{itemize}
\end{frame}

\begin{frame}
  {Eszett-Silben und die anderen \textit{s}}
  \pause
  \centering
  {\footnotesize\textit{Busen}:}\hspace{1em}\scalebox{0.55}{%
    \begin{forest}
      for tree={s sep+=1em}
      [Phonologisches Wort, calign=first
        [Silbe, calign=last
          [Ar., ake
            [b]
          ]
          [Reim
            [Kern, ake
              [uː]
            ]
          ]
        ]
        [Silbe, calign=last
          [Ar., ake, baseline
            [z]
          ]
          [Reim, calign=first
            [Kern, ake
              [ə]
            ]
            [Er., ake
              [n]
            ]
          ]
        ]
      ]
    \end{forest}
  }~\pause\hspace{1.5em}{\footnotesize\textit{Bussen}:}\hspace{1em}\scalebox{0.55}{%
    \begin{forest}
      for tree={s sep+=1em}
      [Phonologisches Wort, calign=first
        [Silbe, calign=last
          [Ar., ake
            [b]
          ]
          [Reim, calign=first
            [Kern, ake
              [ʊ]
            ]
            [Er., ake, name=BusenEr]
          ]
        ]
        [Silbe, calign=last
          [Ar., ake, baseline
            [s]
            {\draw[-] (.north) -- (BusenEr.south);}
          ]
          [Reim, calign=first
            [Kern, ake
              [ə]
            ]
            [Er., ake
              [n]
            ]
          ]
        ]
      ]
    \end{forest}
  }\\
  \pause
  \Zeile
  {\footnotesize\textit{Bußen} mit \alert{Endranddesonorisierung} und \orongsch{Assimilation}:}\hspace{1em}\scalebox{0.55}{%  
    \begin{forest}
      for tree={s sep+=1em}
      [Phonologisches Wort, calign=first
        [Silbe, calign=last
          [Ar., ake
            [b]
          ]
          [Reim, calign=first
            [Kern, ake
              [uː]
            ]
            [Er., ake
              [\alert{\textbf{s}}]
            ]
          ]
        ]
        [Silbe, calign=last
          [Ar., ake, baseline
            [\orongsch{\textbf{s}}]
          ]
          [Reim, calign=first
            [Kern, ake
              [ə]
            ]
            [Er., ake
              [n]
            ]
          ]
        ]
      ]
    \end{forest}
  }
\end{frame}


\begin{frame}
  {Schritt für Schritt}
  \pause
  \begin{enumerate}[<+->]
    \item zugrundeliegende Form: \alert{/buzzən/}
    \item Silbifizierung \phopro \{buz\orongsch{.}zən\}
    \item Längung gespannter Vokale \phopro \{b\orongsch{uː}z.zən\}
    \item Endranddesonorisierung \phopro \{buː\orongsch{s}.zən\}
    \item Assimilation des Anfangsrands \phopro \alert{[buːs.}\orongsch{s}\alert{ən]}
  \end{enumerate}
  \pause
  \begin{itemize}[<+->]
    \item Ist die Assimilation ein Taschenspielertrick?
    \item Nein, denn sie findet auch in anderen Fällen statt!
  \end{itemize}
  \pause
  \begin{exe}
    \ex\label{ex:dehnungsundschaerfungsschreibungen024}
    \begin{xlist}
      \ex{\label{ex:dehnungsundschaerfungsschreibungen025} /ɛ̆k\alert{z}ə/ \phopro\ [ʔɛk.\orongsch{s}ə] (\textit{Echse})}
      \pause
      \ex{\label{ex:dehnungsundschaerfungsschreibungen026} /ɛ̆ʁb\alert{z}e/ \phopro\ [ʔɛ͡əp.\orongsch{s}ə] (\textit{Erbse})}
    \end{xlist}
  \end{exe}
  \pause
  \begin{itemize}[<+->]
    \item Also ist das Konsonantenzeichen \textit{s} \rot{nicht} doppelt belegt.
    \item \alert{Es gibt zugrundeliegend nur /z/.}
  \end{itemize}
\end{frame}


\section[Konstanz]{Konstantschreibung}

\begin{frame}
  {Zur Erinnerung: unerklärte Doppelkonsonanten}
  \pause
  \centering
  \resizebox{0.7\textwidth}{!}{
    \begin{tabular}{lllllllll}
      \toprule
      & & & \textbf{ɪ} & \textbf{ʊ} & \multicolumn{2}{l}{\LocStrutGrph\textbf{ɛ̆}} & \textbf{ɔ} & \textbf{ă} \\
      \midrule

      \multirow{4}{*}{\rotatebox{90}{\textbf{ungespannt}}}

      & \multirow{2}{*}{\rotatebox{90}{\textbf{offen}}}
      & \textbf{einsilb.}  & \textit{\Nono}  & \textit{\Nono}           & \multicolumn{2}{l}{\LocStrutGrph\textit{\Nono}}         & \textit{\Nono}        & \textit{\Nono}           \\
      && \textbf{zweisilb.}  & \textit{Li.\alert{pp}e} & \textit{Fu.\alert{tt}er}         & \multicolumn{2}{l}{\LocStrutGrph\textit{We.\alert{ck}e}}        & \textit{o.\alert{ff}en}       & \textit{wa.\alert{ck}er}         \\
        & \multirow{2}{*}{\rotatebox{90}{\textbf{gesch.}}}
        & \textbf{einsilb.}  & \textit{Ki\rot{nn}}   & \textit{Schu\rot{tt}}    & \multicolumn{2}{l}{\LocStrutGrph\textit{Be\rot{tt}}}           & \textit{Ro\rot{ck}}         & \textit{Wa\rot{tt}}            \\
        && \textbf{zweisilb.}  & \textit{Rin.de} & \textit{Wun.der}        & \multicolumn{2}{l}{\LocStrutGrph\textit{Wen.de}}        & \textit{pol.ter}      & \textit{Tan.te}          \\

      \midrule

      \multirow{4}{*}{\rotatebox{90}{\textbf{gespannt}}}

      & \multirow{2}{*}{\rotatebox{90}{\textbf{offen}}}
        & \textbf{einsilb.}  & \textit{Knie}   & \textit{Schuh}       & \textit{Schnee, Reh}  & \textit{zäh}          & \textit{roh}          & (\textit{da})            \\
      && \textbf{zweisilb.}  & \textit{Bie.ne} & \textit{Kuh.le, Schu.le} & \textit{we.nig}       & \textit{Äh.re, rä.kel} & \textit{oh.ne, O.fen} & \textit{Fah.ne, Spa.ten} \\

      & \multirow{2}{*}{\rotatebox{90}{\textbf{gesch.}}}
        & \textbf{einsilb.}  & \textit{lieb}  & \textit{Ruhm, Glut}      & \textit{Weg}          & \textit{spät}           & \textit{rot}          & \textit{Tat}             \\
      && \textbf{zweisilb.}  & (\textit{lieb.lich}) & (\textit{lug.te})   & (\textit{red.lich})   & (\textit{wähl.te})     & (\textit{brot.los})   & (\textit{rat.los})       \\

      \midrule
      & & & \textbf{i} & \textbf{u} & \textbf{e} & \textbf{ε} & \textbf{o} & \textbf{a} \\

      \bottomrule
    \end{tabular}
  }
\end{frame}

\begin{frame}
  {Lösung | Konstanz}
  \begin{itemize}[<+->]
    \item Warum \textit{Kinn}, \textit{Schutt}, \textit{Bett}, \textit{Rock}, \textit{Wattes}?
      \Halbzeile
    \item \alert{nicht unterlassbare Gelenkschreibungen}
      \begin{itemize}[<+->]
        \item \textit{die Ki\alert{nn}e}
        \item \textit{des Schu\alert{tt}es}
        \item \textit{die Be\alert{tt}en}
        \item \textit{die Rö\alert{ck}e}
      \end{itemize}
      \Halbzeile
    \item \alert{Die Schreibungen eines Stamms einander angleichen!} Sonst:
      \begin{itemize}[<+->]
        \item \textit{*Kin --- Kinne}
        \item \textit{Schut --- Schutt}
        \item \textit{Bet --- Betten}
        \item \textit{Rok --- Röcke}
      \end{itemize}
  \end{itemize}
\end{frame}

\begin{frame}
  {Andere Konstantschreibungen}
  \pause
  \begin{itemize}[<+->]
    \item andere Wortklassen
      \begin{itemize}[<+->]
        \item \textit{*plat --- pla\rot{tt} --- pla\alert{tt}er}
        \item \textit{*as --- a\rot{ß} --- a\alert{ß}en}
        \item aber: \textit{las --- lasen}
        \item \textit{*schlizte --- schli\rot{tz}te --- schli\alert{tz}en}
      \end{itemize}
      \Halbzeile
    \item andere Phänomene (nicht Silbengelenk oder \textit{ß})
      \begin{itemize}[<+->]
        \item \textit{*gest --- ge\rot{h}st --- ge\alert{h}en}
        \item \textit{*siest --- sie\rot{h}st --- se\alert{h}en}
        \item \textit{*Reume --- R\rot{äu}me --- R\alert{au}m}
        \item \textit{*leuft --- l\rot{äu}ft --- l\alert{au}fen}
      \end{itemize}
  \end{itemize}
\end{frame}


\section{Schärfung + Konstanz = überflüssige Dehnung}

\begin{frame}
  {Das Kreuz mit der Dehnungsschreibung}
  \pause
  \begin{itemize}[<+->]
    \item Dehnungs-\textit{h} (\textit{Reh}, \textit{Pfahl}) oder Dehnungs-Doppelvokal (\textit{Saat}, \textit{Boot})
    \item speziell bei \textit{i} (dort fast immer): Dehnungs-\textit{e} (\textit{Knie}, \textit{Dieb})
      \Halbzeile
    \item \alert{weitgehend redundant} (erst recht im Kern)
    \item \alert{unsystematisch} (\textit{Lid}, \textit{Lied} usw.)
      \Halbzeile
    \item mangels Systematik: \alert{oft Erwerbsprobleme}
    \item \ldots\ denen kaum systematisch zu begenen ist
  \end{itemize}
\end{frame}

\begin{frame}
  {Erinnerung | Realisierungen der Dehnungsschreibung}
  \onslide<+->
  \onslide<+->
  Gespanntheitsmarkierung |\\
  \alert{h}, \orongsch{nichts}, \grau{Doppelvokal} oder bei <i> die \alert{<ie>-Schreibung}\\
  \Zeile
  \begin{tabular}{llllll}
    /i/ &           \rot{*<ih>} & \alert{<ie>} & \orongsch{<i>} &          \rot{*<ii>} & \textit{R\alert{ie}men}, \textit{\orongsch{I}gel}, *\textit{Kn\rot{ii}b}, *\textit{Kn\rot{ih}p} \\
    /y/ & \whyte{*}\alert{<üh>} &              & \orongsch{<ü>} &          \rot{*<üü>} & \textit{B\alert{üh}ne}, \textit{m\orongsch{ü}de}, *\textit{B\rot{üü}ke} \\
    /e/ & \whyte{*}\alert{<eh>} &              & \orongsch{<e>} & \whyte{*}\grau{<ee>} & \textit{k\alert{eh}ren}, \textit{w\orongsch{e}nig}, \textit{S\grau{ee}} \\
    /ɛ/ & \whyte{*}\alert{<äh>} &              & \orongsch{<ä>} &          \rot{*<ää>} & \textit{\alert{Äh}re}, \textit{d\orongsch{ä}nisch}, *\textit{S\rot{ää}le} \\
    /ø/ & \whyte{*}\alert{<öh>} &              & \orongsch{<ö>} &          \rot{*<öö>} & \textit{st\alert{öh}nen}, \textit{fl\orongsch{ö}ten}, *\textit{d\rot{öö}fer} \\
    /u/ & \whyte{*}\alert{<uh>} &              & \orongsch{<u>} &          \rot{*<uu>} & \textit{K\alert{uh}le}, \textit{Sch\orongsch{u}le}, *\textit{Kr\rot{uu}fe} \\
    /o/ & \whyte{*}\alert{<oh>} &              & \orongsch{<o>} & \whyte{*}\grau{<oo>} & \textit{L\alert{oh}n}, \textit{B\orongsch{o}den}, \textit{d\grau{oo}f} \\
    /a/ & \whyte{*}\alert{<ah>} &              & \orongsch{<a>} & \whyte{*}\grau{<aa>} & \textit{W\alert{ah}n}, \textit{b\orongsch{a}den}, \textit{\grau{Aa}l} \\
  \end{tabular}\\
  \Zeile 
  \onslide<+->
  <i>, <u> und Umlautgraphen können nicht gedoppelt werden!
\end{frame}


\begin{frame}
  {Redundanz von Dehnungsschreibungen im Kern}
  \alert{Ausnahmslosigkeit der Schärfungsschreibung} und \\
  \alert{Konstanzprinzip} führen zu \rot{Redundanz der Dehnungsschreibung}\\
  \Zeile
  \onslide<+->
  \centering 
  \begin{tabular}{llll}
    \toprule
    \textbf{Graph} & \textbf{Ortho.} & \alert{\textbf{Ohne DS}} & \rot{\textbf{wäre V kurz}} \\
    \midrule
    <ie> & Lied -- Lieder & \alert{Lid -- Lider} & \rot{Lidd -- Lidder} \\
    <üh> & Bühne & \alert{Büne} & \rot{Bünne} \\ 
    <eh> & kehr -- kehren & \alert{ker -- keren} & \rot{kerr -- kerren} \\ 
    <äh> & Ähre & \alert{Äre} & \rot{Ärre} \\ 
    <aa> & Saal -- Säle & \alert{Sal -- Säle} & \rot{Säll -- Sälle} \\
    <öh> & stöhn -- stöhnen & \alert{stön -- stönen} & \rot{stönn -- stönnen} \\
    <uh> & Kuhle & \alert{Kule} & \rot{Kulle} \\ 
    <oh> & Lohn -- Löhne & \alert{Lon -- Löne} & \rot{Lönn -- Lönne} \\ 
    <ah> & Wahn -- Wahnes & \alert{Wan -- Wanes} & \rot{Wann -- Wannes} \\ 
    \bottomrule
  \end{tabular}
\end{frame}

\begin{frame}
  {Kann das weg?}
  \onslide<+->
  \onslide<+->
  \Large \alert{Die Dehnungsschreibung ist\\
  vom System aus gesehen im kern entbehrlich.}\\
  \Halbzeile
  \grau{Und in der Peripherie (vor allem Lehnwortschreibungen)\\
  kommt sie sowieso nicht zum Einsatz.}\\
  \Zeile
  \onslide<+->
  \alert{Sie ist unsystematisch und nicht regelhaft lernbar.}\\
  \Zeile
  \onslide<+->
  \rot{Wir brauchen die Dehnungsschreibung nicht!}
\end{frame}


\ifdefined\TITLE
  \section{Nächste Woche | Überblick}

  \begin{frame}
    {Der ungefähre Semesterplan}
    \begin{enumerate}[<+->]
      \item Graphematik und Schreibprinzipien
      \item Wiederholung -- Phonetik
      \item Wiederholung -- Phonologie
      \item Phonographisches Schreibprinzip -- Konsonanten
      \item Phonographisches Schreibprinzip -- Vokale
      \item Silben und Dehnungsschreibungen
      \item Eszett, Dehnung und Konstanz
      \item \alert{Spatien und Majuskeln}
      \item Komma
      \item Punkt und sonstige Interpunktion
    \end{enumerate}
  \end{frame}
\fi
